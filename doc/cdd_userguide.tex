\documentclass[12pt]{article}
\usepackage[a4paper,margin=3cm,footskip=2cm]{geometry}
\usepackage{lmodern}
\usepackage[utf8]{inputenc}
\usepackage[T1]{fontenc}
\usepackage{textcomp}
\usepackage{verbatim}
\usepackage{enumitem}
\usepackage{longtable}
\usepackage{alltt}
\usepackage{ifthen}
\usepackage[backend=biber, maxbibnames=99, defernumbers, sorting=none]{biblatex}
\addbibresource{PsN.bib}
% Reduce the size of the underscore
\usepackage{relsize}
\renewcommand{\_}{\textscale{.7}{\textunderscore}}

\input{inputs/version.tex}

\newcommand{\revisiondate}[1]{
\date{#1}
}
\newcommand{\guidetitle}[2]{
\title{#1\\ \vspace{2 mm} {\large PsN \psnversion}}
\date{Revised #2}
}

\newcommand{\references}{
    \printbibliography
}

\newcommand{\doctitle}[2]{
\title{#1}
\date{#2}
}


\newenvironment{optionlist}{
\renewcommand{\arraystretch}{1.1}
\setlength{\leftmargini}{2.5cm}
\begin{description}
%\setlength{\itemsep}{0ex}
}
{\end{description}}

\newcommand{\optname}[1]{\item{{\bfseries\texttt-#1}\newline}}
\newcommand{\optdefault}[2]{\item{{\bfseries\texttt-#1}{\mbox{ = \it #2}}\newline}}
% optconfig is for options that can only be set in a (scm) config file, always without -
\newcommand{\optconfig}[2]{\item{{\bfseries\texttt#1}{\mbox{ = \it #2}}\newline}}

\newcommand{\nextopt}{}


\hyphenation{NONMEM}
\hyphenation{INPUT}

\guidetitle{CDD user guide}{2016-04-13}


\begin{document}

\maketitle
\newcommand{\guidetoolname}{cdd}


\section{Introduction}
The Case Deletions Diagnostics (CDD) algorithm is a tool primarily used to identify influential components of the dataset, usually individuals. The CDD works by identifying groups in the data set and creating one new data set for each member of the group, where that member has been removed. The model is run once with each new data set. The PsN implementation of the CDD can take any column as base for the grouping and all rows with the same value in that column will be considered a group as long as no individual contain multiple values in that column.
One should take care that the grouping creates sensible individual records. PsN will renumber the ID column so that two individuals with the same ID will not end up next to each other.
Examples
\begin{verbatim}
cdd moxonidine.mod -case_column=1
cdd pheno.mod -case_column=AGE
\end{verbatim}

\section{Input and options}

\subsection{Required input}
A model file is required on the command-line.

\begin{optionlist}
\optdefault{case\_column}{name|number}
The column on which the case-deletion is done. You can either use the name of the column as specified in the \mbox{\$INPUT} record in the model file or you can use the column number in the \mbox{\$INPUT} record. Numbering starts with 1.

\nextopt
\end{optionlist}

\subsection{Optional input}

\begin{optionlist}
\optdefault{bins}{N}
Sets the number of databins, or cdd datasets, to use. If the number of unique values, or factors, in the case column is higher than N then one or more factors will be deleted in each cdd dataset. Specifying N as higher than the number of factors will have no effect. N is then reset to the number of factors. Default value = Number of unique values in the case column. 
\nextopt
\optname{xv}
Default true. Run the cross-validation step (-xv) or not (-no-xv). 
\nextopt
\optdefault{selection\_method}{random | consecutive}
Default consecutive. Specifies whether the factors selected for exclusion should be drawn randomly or consecutively from the datafile. 
\nextopt
\optdefault{outside\_n\_sd\_check}{X}
Default 2. Mark the runs with CS-CR outside X standard deviations of the PCA. 
\nextopt
\end{optionlist}

\subsection{Some important common PsN options}
There are many options that govern how PsN manages NONMEM runs, and
those options are common to all PsN programs that run NONMEM.
For a complete list see common\_options.pdf, 
or psn\_options -h on the commandline.
\begin{optionlist}
\optname{h or -?}
Print the list of available options and exit. 
\nextopt
\optname{help}
With -help all programs will print a longer help message. 
If an option name is given as argument, help will be printed for this option. 
If no option is specified, help text for all options will be printed. 
\nextopt
\optdefault{directory}{'string'}
Default \guidetoolname\_dirN,
where N will start at 1 and
be increased by one each time you run the script. The directory option sets the directory in which PsN 
will run NONMEM and where PsN-generated output files will be stored. 
You do not have to create the directory,  it will be done for you. If you set
-directory to a the name of a directory that already exists, PsN will run in the existing directory.
\nextopt
\optdefault{seed}{'string'}
You can set your own random seed to make PsN runs reproducible.
The random seed is a string, so both -seed=12345 and -seed=JustinBieber are valid.
It is important to know that because of the way the Perl pseudo-random
number generator works, for two similar string seeds the random sequences may be identical. 
This is the case e.g. with the two different seeds 123 and 122. 
Setting the same seed guarantees the same sequence, but setting two slightly different 
seeds does not guarantee two different random sequences, that must be verified.
\nextopt
\optdefault{clean}{'integer'}
Default 1. The clean option can take four different values:  
\begin{description}
\item[0] Nothing is removed 
\item[1] NONMEM binary and intermediate files except INTER are removed, and files specified with option -extra\_files. 
\item[2] model and output files generated by PsN restarts are removed, and data files in the NM\_run directory, and (if option -nmqual is used) the xml-formatted NONMEM output. 
\item[3] All NM\_run directories are completely removed. If the PsN tool has created modelfit\_dir:s inside the main run directory, these  will also be removed. 
\end{description}
\nextopt
\optname{zip}
If this option is set the m1 folder in the run directory will be zipped into the file m1.zip and removed. This will save space and reduce the number of files generated by PsN. PsN can seamlessly handle zipped m1 folders for example when rerunning a command in the same directory using the -directory option.
\nextopt
\optdefault{nm\_version}{'string'}
Default is 'default'. 
If you have more than one NONMEM version installed you can use option
-nm\_version to choose which one to use, as long as it is 
defined in the [nm\_versions] section in psn.conf, see psn\_configuration.pdf for details. 
You can check which versions are defined, without opening psn.conf, using the command
\begin{verbatim}
psn -nm_versions
\end{verbatim}
\nextopt
\optdefault{threads}{'integer'}
Default 5 (if default PsN config file is used). 
Use the threads option to enable parallel execution of multiple models.
This option decides how many models PsN will run at the same time, and it is completely
independent of whether the individual models are run with serial NONMEM or parallel NONMEM.
If you want to run a single model in parallel you must use options -parafile and -nodes.
On a desktop computer it 
is recommended to not set -threads higher the number of CPUs in the system plus one. 
You can specify more threads, 
but it will probably not increase the performance. If you are running on a computer cluster, 
you should consult your 
system administrator to find out how many threads you can specify. 
\nextopt
\optname{version}
Prints the PsN version number of the tool, and then exit. 
\nextopt
\optname{citations}
Print a list of references for this tool. The list will be in BibTeX format.
\nextopt
\end{optionlist}

\begin{optionlist}
\optname{last\_est\_complete}
is optional and only applies with NONMEM7 and cdd option -xv. 
See common\_options.pdf for details.
\nextopt
\end{optionlist}

\subsection{cdd rplots}
\newcommand{\rplotsconditions}{
If option -rplots is set $>=1$, a plot with Covariance ratios
vs Cook scores for each case, e.g. ID, will be generated. 
The default cdd rplots template 
requires no special R libraries.
If no pdf is generated,
see the .Rout file in the main run directory for error messages.}

PsN can automatically generate R plots to visualize results, if a template file is available.
The PsN installation package includes default template files for a number of tools,
but R plots can be generated for any tool if the user provides the template file.
PsN will create a preamble with some run specific information, 
such as the name of the raw\_results file, the parameter labels and the name of
the tool results file,
and then append 
the template file.
If the template file contains an old preamble, then that preamble will be replaced with 
the new one. This means
the user can modify an old PsN-generated R script and then use this script as a new template,
without having to remove the preamble.
When a template file is available for the tool and option \mbox{-rplots} is 0 or positive, 
the R script will be generated and saved in the main
run directory. 
If R is configured in psn.conf or command 'R' is available and option -rplots is positive 
the script will also be run and a number of pdf-format plots be created.

\rplotsconditions

\begin{optionlist}
\optdefault{rplots}{level}
-rplots<0 means R script is not generated\\ 
-rplots=0 (default) means R script is generated but not run\\ 
-rplots=1 means basic plots are generated\\													  
-rplots=2 means basic and extended plots are generated\\													  
\nextopt
\optdefault{template\_file\_rplots}{file}
When the rplots feature is used, the default template file PsN will use is 
\guidetoolname\_default.R. 
The user can choose a different template file
by setting option -template\_file\_rplots to a different file. 
PsN will first look for the file relative to the current working directory, 
and after that in the -template\_directory\_rplots directory.
\nextopt
\optdefault{template\_directory\_rplots}{path}
PsN can look for the rplots template file in a number of places. The priority order is the
following:
\begin{enumerate}
\item template\_directory\_rplots set on command-line 
\item calling directory (where PsN is started)
\item template\_directory\_rplots set in psn.conf 
\item R-scripts subdirectory of the PsN installation directory
\end{enumerate}
\nextopt
\optdefault{subset\_variable\_rplots}{variable name}
Default not set. The user can specify a subset variable to be used with the -rplots feature. 
This variable
will, if set, be set as subset.variable in the preamble,
and can then be used in the plotting code. 
\nextopt
\end{optionlist}



\section{Known bugs/issues}

If NONMEM6 is used with the cdd and the S matrix is algorithmically singular (message in lst-file, checked also by sumo script) the Cook scores cannot  be trusted. The cdd does not check for this error. 

\end{document}
