\documentclass[12pt]{article}
\usepackage[a4paper,margin=3cm,footskip=2cm]{geometry}
\usepackage{lmodern}
\usepackage[utf8]{inputenc}
\usepackage[T1]{fontenc}
\usepackage{textcomp}
\usepackage{verbatim}
\usepackage{enumitem}
\usepackage{longtable}
\usepackage{alltt}
\usepackage{ifthen}
\usepackage[backend=biber, maxbibnames=99, defernumbers, sorting=none]{biblatex}
\addbibresource{PsN.bib}
% Reduce the size of the underscore
\usepackage{relsize}
\renewcommand{\_}{\textscale{.7}{\textunderscore}}

\input{inputs/version.tex}

\newcommand{\revisiondate}[1]{
\date{#1}
}
\newcommand{\guidetitle}[2]{
\title{#1\\ \vspace{2 mm} {\large PsN \psnversion}}
\date{Revised #2}
}

\newcommand{\references}{
    \printbibliography
}

\newcommand{\doctitle}[2]{
\title{#1}
\date{#2}
}


\newenvironment{optionlist}{
\renewcommand{\arraystretch}{1.1}
\setlength{\leftmargini}{2.5cm}
\begin{description}
%\setlength{\itemsep}{0ex}
}
{\end{description}}

\newcommand{\optname}[1]{\item{{\bfseries\texttt-#1}\newline}}
\newcommand{\optdefault}[2]{\item{{\bfseries\texttt-#1}{\mbox{ = \it #2}}\newline}}
% optconfig is for options that can only be set in a (scm) config file, always without -
\newcommand{\optconfig}[2]{\item{{\bfseries\texttt#1}{\mbox{ = \it #2}}\newline}}

\newcommand{\nextopt}{}


\hyphenation{NONMEM}
\hyphenation{INPUT}

\guidetitle{BENCHMARK user guide}{2017-09-26}


\begin{document}

\maketitle
\newcommand{\guidetoolname}{benchmark}


\section{Introduction}
The benchmark tool is intended for systematic comparison of run times, ofv and parameter estimates
across NONMEM versions and/or alternative settings of NONMEM options.
For each model file given on the command-line, PsN will create one copy for each possible combination
of the modifications listed on the command-line (option -record\_options), and
then each model copy will be run first with the nonmem version given
with option -nm\_version (as with any PsN tool) and then also with
each of the nonmem versions listed with option -alt\_nonmem (if any).
Each model run will be repeated if option -replicates is set larger than 1.

There will be one raw\_results file produced for each input model, and each raw\_results file will contain 
'replicates' row(s)
for each combination of model modifications and NONMEM version. 
%The rplots functionality is used to
%read the raw\_results files and
%visualize the run times (as reported by NONMEM in the lst-files), the stability of ofv values and of estimates.

Note: The run times reported will be estimation step times, covariance step times and total times reported in the NONMEM lst-files.
PsN will not measure times, only parse the contents of the lst-file. It is up to the user to ensure that the run conditions 
(hard-ware, cpu load) are comparable across NONMEM runs (e.g. run on a homogeneous cluster where queueing system ensures 
no interference between NONMEM runs, or run with -threads=1 on standalone computer and no other significant runs at the same time).

Examples:
\begin{verbatim}
benchmark run1.mod run2.mod -nm_version=nm74alpha 
-record_options=estim:none,FAST -replicates=3

benchmark run3.mod -nm_version=nm73gfortran4 
-alt_nonmem=nm73gfortran5 -merge_rawreults -replicates=2
\end{verbatim}
See below for explanation of each option.

\section{Input and options}

\subsection{Required input}
A model file is required on the command-line. 
Also, there must additional options resulting in at least two NONMEM runs that can be compared. This can be
achieved by setting at least one alternative NONMEM version or at least one record modification listed,
or giving at least two model files as input and setting option -merge\_rawresults. 

\subsection{Optional input}

\begin{optionlist}
\optdefault{alt\_nonmem}{list}
Default not set. 
A comma-separated list of NONMEM versions, as named in psn.conf to run all, possibly modified, models with
\emph{in addition to} the NONMEM version set with PsN common option -nm\_version.
If different compilers for the same NONMEM version are to be compared, 
then the same NONMEM version compiled with different compilers
must be listed with different names in psn.conf, since PsN does not ''know'' about compilers.
\nextopt
\optname{copy\_data}
Default set. Disable with -no-copy\_data. By default, PsN will copy the datafile into NM\_run1 and set a local path in psn.mod, 
the actual modelfile run with NONMEM. If -no-copy\_data is set, PsN will not copy the data to NM\_run1 and instead set a 
global path to the datafile in psn.mod. However, NONMEM will not accept a path longer than 80 characters.
\nextopt
%\optdefault{dofv\_threshold}{number}
%Default 1 (percent). If the relative difference between the ofv of two model runs in the same raw\_results file
%is greater than this threshold, these models will be flagged in an extra column in the raw\_results file.
%\nextopt
%\optdefault{parameter\_threshold}{number}
%Default 5 (percent). If the relative difference between the parameter estimates of two model runs 
%in the same raw\_results file
%is greater than this threshold, these models will be flagged in an extra column in the raw\_results file.
%\nextopt
\nextopt
\optname{merge\_rawresults}
Default not set. The default is to create a separate raw\_results file for each input model,
where results for all the variants resulting from -record\_options and -alt\_nonmem
are collected. Creating separate raw\_results files for each input model is done because
it cannot be assumed that the models and sets of population parameters are the same and comparable,
so an investigation of the stability of e.g. parameter estimates across different input models
is not meaningful and the headers in the raw\_results file can be completely wrong for some models. 
However, the user can choose to override the default procedure
with option -merge\_rawresults, resulting in all runs being reported in the same raw\_results file,
and the rplots processing being performed for all models as a single group.
\nextopt
\optdefault{record\_options}{list}
Default not set.
A structured list of the control stream records to modify, and the options to set. 
The record names can be abbreviated and written in lower or upper case,
as long as they are correct abbreviations of the complete record names. 
For example: estimation, EST and estim are correct while ESIMATION
is wrong. After the record name there must be colon (:), and then
a comma-separated list of two or more alternative options.
If only one alternative is set then this modification will be made to all models, which only makes
sense if there are other sets of alternative modifications.
The option names must be in upper case, unless ``none'' is specified, which is 
a marker for ``no modification of the record''. Example: \\
-record\_options=estimation:none, FAST means that
each model should be run both with option FAST added to \$ESTIMATION and no modification to 
\$ESTIMATION.
If more than one record-options combination is desired, the different combinations are 
separated with double comma (,,). Example: \\
-record\_options=est:none,FAST,,est:ATOL=10,ATOL=8,ATOL=12 
which would result in six (2 times 3) variants of each model.
\nextopt
\optdefault{reference\_lst}{list}
Add the results from one or more pre-existing .lst files to all raw\_results.
Saves time if reference results already available, e.g. when comparing results
from old NONMEM versions to a newly installed version.
\nextopt
\optdefault{replicates}{N}
Default 1. Can be set larger than 1 e.g. for run time comparisons.
See note about user responsibility in ensuring run times are comparable.
\nextopt
\optdefault{theta\_inits}{list}
Default not set.
A structured list of the \$THETA initial estimates to modify. If used, the 
\$THETAs in question will be modified regardless of whether they are FIXED in the control stream.
In the list, the \$THETAs are identified by their order number, where numbering starts at 1,
or by their PsN format label, as CL in this example:
\begin{verbatim}
$THETA (0,1.005) ; CL
\end{verbatim}
In the list, after the theta label/number there must be a colon and then a comma-separated list of numbers
to set as initial estimates. If 'none' is set it means do not change the input model.
If an additional theta is to be modified, then a double comma separates 
the two sub-lists. If no \$THETA pre-exists with the label given in the list,
a \$THETA will be added with that initial estimate and label. 
If \$THETA number is used but no \$THETA exists with that number then benchmark will halt with an
error message.
If the same theta is referred to with both label and theta number, this will not be detected
by benchmark and will result in incorrect results.
Example, using both label and theta number
\begin{verbatim}
-theta_inits=CL:none,1.5,1.0,,5:8.5,12
\end{verbatim}

\end{optionlist}

\subsection{Some important common PsN options}
There are many options that govern how PsN manages NONMEM runs, and
those options are common to all PsN programs that run NONMEM.
For a complete list see common\_options.pdf, 
or psn\_options -h on the commandline.

The -threads option, see below, must be set carefully in benchmark if the run time
of models is to be investigated. Comparison of run times will be meaningless
if the individual NONMEM runs are not performed under the same conditions.
\begin{optionlist}
\optname{h or -?}
Print a list of available options and exit. 
\nextopt
\optname{help}
With -help all programs will print a longer help message. If an option name is given as argument, help will be printed for this option. If no option is specified, help text for all options will be printed. 
\nextopt
\optdefault{clean}{integer}
Default is 1. The clean option can take six different values:  
\begin{description}
	\item[0] Nothing is removed 
	\item[1] NONMEM binary and intermediate files except INTER are removed, and files specified with option -extra\_files. 
	\item[2] model and output files generated by PsN restarts are removed, and data files in the NM\_run directory. 
	\item[3] All NM\_run directories are completely removed. If the PsN tool has created modelfit\_dir:s inside the main run directory, these  will also be removed. 
	\item[4] All NM\_run directories and all m1 directories are completely removed.
    \item[5] The entire run directory is removed. This is only useful for execute. The lst-file will be copied even if the run failed.
\end{description}
\nextopt
\optdefault{directory}{string}

Default \guidetoolname\_dirN,
where N will start at 1 and be increased by one each time you run the script. The directory option sets the directory in which PsN 

will run NONMEM and where PsN-generated output files will be stored. You do not have to create the directory,  it will be done for you. If you set -directory to a the name of a directory that already exists, PsN will run in the existing directory, except for scm, boot\_scm and xv\_scm that cannot be started in an existing directory.
\nextopt
\optname{model\_subdir}
	Use an alternative directory structure for PsN. An extra directory
    level unique to each model is introduced between the calling
    directory and the rundirectory. More information about this option can
    be found in PsN.pdf.
\nextopt
   
\optdefault{nm\_version}{string}
Default is 'default'. 
If you have more than one NONMEM version installed you can use option -nm\_version to choose which one to use, as long as it is 

defined in the [nm\_versions] section in psn.conf, see psn\_configuration.pdf for details. You can check which versions are defined, without opening psn.conf, using the command

\begin{verbatim}
psn -nm_versions
\end{verbatim}
\nextopt
\optdefault{seed}{string}
You can set your own random seed to make PsN runs reproducible. The random seed is a string, so both -seed=12345 and -seed=JustinBieber are valid. It is important to know that because of the way the Perl pseudo-random number generator works, for two similar string seeds the random sequences may be identical. This is the case e.g. with the two different seeds 123 and 122. From limited tests it seems as if the final character is ignored and a work around to be sure to set different seeds would be to add a dummy final character. 
Setting the same seed guarantees the same sequence, but setting two slightly different seeds does not guarantee two different random sequences, that must be verified.
\nextopt

\optdefault{threads}{integer}

Default is 5 (if the default psn.conf is used). Use the threads option to enable parallel execution of multiple models.

This option decides how many models PsN will run at the same time, and it is completely independent of whether the individual models are run with serial NONMEM or parallel NONMEM. If you want to run a single model in parallel you must use options -parafile and -nodes. On a desktop computer it is recommended to not set -threads higher the number of CPUs in the system plus one. 
You can specify more threads, but it will probably not increase the performance. If you are running on a computer cluster, you should consult your system administrator to find out how many threads you can specify. 
\nextopt
\optname{version}
Prints the PsN version number of the tool, and then exit. 
\nextopt
\end{optionlist}

\begin{optionlist}
\optname{prepend\_model\_file\_name}
Default not set. If want to keep and compare table files.
\end{optionlist}


\section{Output}
The output from benchmark is
%, in addition to the plots,
one raw\_results file for each input model, or all input models together if -merge\_rawresults was set.
In each file the first model run will be used as reference when computing differences in output.
The raw\_results file(s) will have extra columns for the input model name, each
modification set via -record\_options, the NONMEM version if -alt\_nonmem was used, and the replicate
number if -replicates was set.
%\begin{itemize}
%\item delta-ofv (dofv) relative to the reference run
%\item delta-runtime 
%\item delta-estimates 
%\end{itemize}


\end{document}

\subsection{Auto-generated R-plots from PsN}
\newcommand{\rplotsconditions}{The default benchmark template 
requires no special R libraries.
If no pdf is generated,
see the .Rout file in the main run directory for error messages.}
PsN can automatically generate R plots to visualize results for \guidetoolname, using a default template found in the R-scripts subdirectory of the installation directory. The user can also create a custom template, see more details in the section Auto-generated R plots from PsN in common\_options.pdf.

\rplotsconditions

\begin{optionlist}
\optdefault{rplots}{level}
-rplots<0 means R-script is not generated\\ 
-rplots=0 (default) means R-script is generated but not run\\ 
-rplots=1 means basic R plots are generated\\													  
-rplots=2 means basic and extended R plots are generated\\													  
\nextopt
\end{optionlist}

\subsubsection*{Troubleshooting}
If no .pdf was generated even if a template file is available and the appropriate options were set, check the .Rout-file in the main run directory for error messages. If no .Rout-file exists, then check that R is properly installed, and that either command 'R' is available or that R is configured in psn.conf.


\subsubsection*{Basic plots}
Basic diagnostic rplots for each input model, or all input models together if -merge\_rawresults was set, 
will be generated if option -rplots is set >0.
The basic plots show differences in total run time, estimation time, covariance time and ofv.
If the ofv-difference between any two runs is greater than a threshold (see option -dofv\_threshold), this is flagged. 

\subsubsection*{Extended plots}
Extended diagnostic rplots for each input model, or all input models together if -merge\_rawresults was set, 
will be generated if option -rplots is set >1.
The extended plots show differences in parameter estimates and standard errors.
If the estimate difference is greater than a threshold (see option -parameter\_threshold) then this is flagged.
