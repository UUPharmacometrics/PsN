\documentclass[12pt]{article}
\usepackage[a4paper,margin=3cm,footskip=2cm]{geometry}
\usepackage{lmodern}
\usepackage[utf8]{inputenc}
\usepackage[T1]{fontenc}
\usepackage{textcomp}
\usepackage{verbatim}
\usepackage{enumitem}
\usepackage{longtable}
\usepackage{alltt}
\usepackage{ifthen}
\usepackage[backend=biber, maxbibnames=99, defernumbers, sorting=none]{biblatex}
\addbibresource{PsN.bib}
% Reduce the size of the underscore
\usepackage{relsize}
\renewcommand{\_}{\textscale{.7}{\textunderscore}}

\input{inputs/version.tex}

\newcommand{\revisiondate}[1]{
\date{#1}
}
\newcommand{\guidetitle}[2]{
\title{#1\\ \vspace{2 mm} {\large PsN \psnversion}}
\date{Revised #2}
}

\newcommand{\references}{
    \printbibliography
}

\newcommand{\doctitle}[2]{
\title{#1}
\date{#2}
}


\newenvironment{optionlist}{
\renewcommand{\arraystretch}{1.1}
\setlength{\leftmargini}{2.5cm}
\begin{description}
%\setlength{\itemsep}{0ex}
}
{\end{description}}

\newcommand{\optname}[1]{\item{{\bfseries\texttt-#1}\newline}}
\newcommand{\optdefault}[2]{\item{{\bfseries\texttt-#1}{\mbox{ = \it #2}}\newline}}
% optconfig is for options that can only be set in a (scm) config file, always without -
\newcommand{\optconfig}[2]{\item{{\bfseries\texttt#1}{\mbox{ = \it #2}}\newline}}

\newcommand{\nextopt}{}


\hyphenation{NONMEM}
\hyphenation{INPUT}

\guidetitle{NPFIT user guide}{2017-12-11}

\newcommand{\guidetoolname}{npfit}

\begin{document}

\maketitle


\section{Introduction}
Running a modelfile with \$NONPARAMETRIC UNCONDITIONAL and different values of NPSUPP.
Example:
\begin{verbatim}
npfit run1.mod -npsupp=50,100,200
\end{verbatim}

\section{Input and options}

\subsection{Required input}
A model file is required, as well as a list of npsupp-values. NONMEM requires that the \$ESTIMATION record is present with the conditional method or POSTHOC.
\begin{optionlist}
\optdefault{npsupp}{50,100,200}
A comma-separated list of non-negative integers. For each value N a new copy of the input model will be run with \$NONPARAMETRIC UNCONDITIONAL NPSUPP=N.\\ See the NONMEM documentation on \$NONPARAMETRIC for interpretation of NPSUPP.
All values in the list should be equal to or greater than the number of individuals in the data set.
\nextopt
\end{optionlist}
%\subsection{Optional input}



%\optname{copy\_data}
%\nextopt
%\optname{keep\_tables}
%\nextopt
%\optdefault{rplots}{level}
%\nextopt


\subsection{Some important common PsN options}
There are many options that govern how PsN manages NONMEM runs, and
those options are common to all PsN programs that run NONMEM.
For a complete list of such options see common\_options.pdf, 
or psn\_options -h on the commandline. A selection of
the most important common options relevant for npfit is found here.
\begin{optionlist}
\optname{h or -?}
Print a list of available options and exit. 
\nextopt
\optname{help}
With -help all programs will print a longer help message. If an option name is given as argument, help will be printed for this option. If no option is specified, help text for all options will be printed. 
\nextopt
\optdefault{clean}{integer}
Default is 1. The clean option can take six different values:  
\begin{description}
	\item[0] Nothing is removed 
	\item[1] NONMEM binary and intermediate files except INTER are removed, and files specified with option -extra\_files. 
	\item[2] model and output files generated by PsN restarts are removed, and data files in the NM\_run directory. 
	\item[3] All NM\_run directories are completely removed. If the PsN tool has created modelfit\_dir:s inside the main run directory, these  will also be removed. 
	\item[4] All NM\_run directories and all m1 directories are completely removed.
    \item[5] The entire run directory is removed. This is only useful for execute. The lst-file will be copied even if the run failed.
\end{description}
\nextopt
\optdefault{directory}{string}

Default \guidetoolname\_dirN,
where N will start at 1 and be increased by one each time you run the script. The directory option sets the directory in which PsN 

will run NONMEM and where PsN-generated output files will be stored. You do not have to create the directory,  it will be done for you. If you set -directory to a the name of a directory that already exists, PsN will run in the existing directory, except for scm, boot\_scm and xv\_scm that cannot be started in an existing directory.
\nextopt
\optname{model\_subdir}
	Use an alternative directory structure for PsN. An extra directory
    level unique to each model is introduced between the calling
    directory and the rundirectory. More information about this option can
    be found in PsN.pdf.
\nextopt
   
\optdefault{nm\_version}{string}
Default is 'default'. 
If you have more than one NONMEM version installed you can use option -nm\_version to choose which one to use, as long as it is 

defined in the [nm\_versions] section in psn.conf, see psn\_configuration.pdf for details. You can check which versions are defined, without opening psn.conf, using the command

\begin{verbatim}
psn -nm_versions
\end{verbatim}
\nextopt
\optdefault{seed}{string}
You can set your own random seed to make PsN runs reproducible. The random seed is a string, so both -seed=12345 and -seed=JustinBieber are valid. It is important to know that because of the way the Perl pseudo-random number generator works, for two similar string seeds the random sequences may be identical. This is the case e.g. with the two different seeds 123 and 122. From limited tests it seems as if the final character is ignored and a work around to be sure to set different seeds would be to add a dummy final character. 
Setting the same seed guarantees the same sequence, but setting two slightly different seeds does not guarantee two different random sequences, that must be verified.
\nextopt

\optdefault{threads}{integer}

Default is 5 (if the default psn.conf is used). Use the threads option to enable parallel execution of multiple models.

This option decides how many models PsN will run at the same time, and it is completely independent of whether the individual models are run with serial NONMEM or parallel NONMEM. If you want to run a single model in parallel you must use options -parafile and -nodes. On a desktop computer it is recommended to not set -threads higher the number of CPUs in the system plus one. 
You can specify more threads, but it will probably not increase the performance. If you are running on a computer cluster, you should consult your system administrator to find out how many threads you can specify. 
\nextopt
\optname{version}
Prints the PsN version number of the tool, and then exit. 
\nextopt
\end{optionlist}


\section{Results}

The file raw\_nonparametric\_modelname.csv, contains the parametric and nonparametric ofv, the npsupp value used, and the nonparametric ETAs.

\section{Internal npfit workflow}

\begin{enumerate}
\item Read the input model into memory.
\item Input checking, i.e. verify that required options are set and valid and that requirements on the input model
are satisfied.
\begin{enumerate}
\item The NONMEM version must be 7.4 or later.
\item The model must have \$ESTIMATION in at least one \$PROBLEM. The first \$PROBLEM with \$ESTIMATION will be the
only one being updated in the following procedure.
\item If METHOD is not conditional then \$ESTIMATION must specify POSTHOC.
\end{enumerate}
\item Create (or reopen) a run directory according to the usual PsN conventions.
\item If the run directory already existed, read any models/results that are already present and skip
to the step below where the old process stopped.
\item If there is an lst-file linked to the input model then read the estimation results.
\item If there are no estimation results available for the input model then run the input model and read the estimation results.
\item Find $N_{ind}$, the number of individuals in the data set as reported in the estimation lst-file,
i.e. the number of individuals after any IGNORE/ACCEPT. If $N_{ind}$ is larger than any of the npsupp values
then print a warning.
\item Update initial estimates with the final estimates from the estimation.
\item If the estimation method is classical set MAXEVAL=1 in \$ESTIMATION. 
\item For each value i of npsupp, copy the updated model and set \$NONPARAMETRIC UNCONDITIONAL NPSUPP=i. Any pre-existing\\ \$NONPARAMETRIC will be removed.
Write the model copy in the m1 subfolder of the run directory.
\item Run models with NONMEM.
\item In the raw\_nonparametric results file, add a column with the npsupp values. 
\item Create, and possibly run, R script.
\end{enumerate}

%\references

\end{document}
