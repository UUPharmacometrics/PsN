
PsN can automatically generate R plots to visualize results for 
\guidetoolname, using a default template found in the R-scripts subdirectory of the installation directory.
The user can also create a custom template, see more details in section 
``Auto-generated R-plots from PsN'' in common\_options.pdf.
When option \mbox{-rplots} is 0 or positive, 
the R script will be generated and saved in the main
run directory. 
If R is configured in psn.conf or command 'R' is available and option -rplots is positive 
the script will also be run and a number of pdf-format plots be created.

\rplotsconditions

\begin{optionlist}
\optdefault{rplots}{level}
-rplots<0 means R script is not generated\\ 
-rplots=0 (default) means R script is generated but not run\\ 
-rplots=1 means basic plots are generated\\													  
-rplots=2 means basic and extended plots are generated\\													  
\nextopt
\optdefault{template\_file\_rplots}{file}
When the rplots feature is used, the default template file PsN will look for is 
\guidetoolname\_default.R. 
The user can choose a different template file
by setting option -template\_file\_rplots to a different file name. 
PsN will look for the file in the 'template\_directory\_rplots' directory, see the help text 
for that option.
\nextopt
\optdefault{template\_directory\_rplots}{path}
PsN can look for the rplots template file in a number of places. 
PsN looks in the following places in the order they are listed:
\begin{enumerate}
\item template\_directory\_rplots from command-line, if set 
\item calling directory (where PsN run is started from)
\item template\_directory\_rplots set in psn.conf 
\item R-scripts subdirectory of the PsN installation directory
\end{enumerate}
\nextopt
\end{optionlist}

\subsubsection*{Troubleshooting}
If no pdf was generated even if a template file is available and the appropriate options
were set, check the .Rout-file in the main run directory for error messages.
