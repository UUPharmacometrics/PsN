\documentclass[12pt]{article}
\usepackage[a4paper,margin=3cm,footskip=2cm]{geometry}
\usepackage{lmodern}
\usepackage[utf8]{inputenc}
\usepackage[T1]{fontenc}
\usepackage{textcomp}
\usepackage{verbatim}
\usepackage{enumitem}
\usepackage{longtable}
\usepackage{alltt}
\usepackage{ifthen}
\usepackage[backend=biber, maxbibnames=99, defernumbers, sorting=none]{biblatex}
\addbibresource{PsN.bib}
% Reduce the size of the underscore
\usepackage{relsize}
\renewcommand{\_}{\textscale{.7}{\textunderscore}}

\input{inputs/version.tex}

\newcommand{\revisiondate}[1]{
\date{#1}
}
\newcommand{\guidetitle}[2]{
\title{#1\\ \vspace{2 mm} {\large PsN \psnversion}}
\date{Revised #2}
}

\newcommand{\references}{
    \printbibliography
}

\newcommand{\doctitle}[2]{
\title{#1}
\date{#2}
}


\newenvironment{optionlist}{
\renewcommand{\arraystretch}{1.1}
\setlength{\leftmargini}{2.5cm}
\begin{description}
%\setlength{\itemsep}{0ex}
}
{\end{description}}

\newcommand{\optname}[1]{\item{{\bfseries\texttt-#1}\newline}}
\newcommand{\optdefault}[2]{\item{{\bfseries\texttt-#1}{\mbox{ = \it #2}}\newline}}
% optconfig is for options that can only be set in a (scm) config file, always without -
\newcommand{\optconfig}[2]{\item{{\bfseries\texttt#1}{\mbox{ = \it #2}}\newline}}

\newcommand{\nextopt}{}


\hyphenation{NONMEM}
\hyphenation{INPUT}

\guidetitle{MCMP user guide}{2015-09-16}


\begin{document}

\maketitle
\newcommand{\guidetoolname}{mcmp}


\section{Introduction}
MCMP – Monte-Carlo Mapped Power \cite{Vong}  – is a tool for power computations.
The method is based on the use of individual Objective Function Values (iOFV) 
and aims to provide a fast and accurate prediction of the power and sample size 
relationship without any need for adjustment of the significance criterion.
The principle of the method is as follows:
\begin{enumerate}
\item a large dataset (e.g. 1000 individuals) is simulated with a full model, or a separate simulation model if model misspecification will be taken into account.
\item the full and reduced models are re-estimated with this data set
\item $iOFV$:s are extracted from the NONMEM .phi-files, and for each subject the difference in $iOFV$ between the full and reduced model is computed ($\Delta iOFV$)
\item $\Delta iOFV$:s are sampled according to the design for which power is to be calculated and a starting sample size (N)
\item the $\Delta iOFV$ sum for each sample is calculated ($\sum_i{\Delta iOFV}$) 
\item steps 4 and 5 are repeated many times 
\item the percentage of $\sum_i{\Delta iOFV}$ greater than the significance criterion (e.g. 3.84 for one degree of freedom and $\alpha=0.05$) 
is taken as the power for sample size N, 
\item steps 4-7 are repeated with increasing N to provide the power at all sample sizes of interest.
\end{enumerate}

Example\\
\begin{verbatim}
mcmp -full_model=mod1.mod -stratify_on=DOSE -reduced_model=mod2.mod -seed=123
\end{verbatim}

\section{Input and options}

\subsection{Mcmp-specific input}
\begin{optionlist}
\optdefault{simulation\_model}{sim.mod}
The filename of the simulation model, optional. 
Default is the -full\_model filename. PsN will modify this model, removing \$EST and adding \$SIM, if it is not already a simulation model.
If both -table\_reduced and -table\_full are specified, then option -simulation\_model will be ignored. Cannot be used with option -simdata.
\nextopt
\optdefault{full\_model}{full.mod}
The filename of the full model is required, unless -table\_full (see below) is used. This must be an estimation model.
\nextopt
\optdefault{reduced\_model}{red.mod}
The name of the reduced model is required, unless -table\_reduced (see below) is used. This must be an estimation model.
\nextopt
\optdefault{stratify\_on}{column}
The name of the variable to stratify on, optional. Must be all uppercase and (unless NM7) at most 4 characters. 
Unless -table\_strata is specified, the variable must be possible to request in \$TABLE, i.e. either present in \$INPUT or defined in \$PK/\$PRED/\$ERROR. PsN does not check that the variable is defined. If -reduced\_model is specified then PsN will set \$TABLE there, otherwise \$TABLE will be set in the full model. The stratification table will not be generated from the simulation model. PsN will set \$TABLE ID $\langle$stratify\_on$\rangle$ FIRSTONLY NOAPPEND NOPRINT ONEHEADER FILE=strata.tab 
\nextopt
\optname{curve}
Set by default. Can be disabled with -no-curve. This option controls whether the complete power curve up to the target power should be generated, or if the program should only compute the sample size required to achieve the target power. 
\nextopt
\optdefault{start\_size}{N}
First total sample size (sum of samples from all strata) to test. Optional, default is 3*increment (see below). 
\nextopt
\optdefault{increment}{N}
Optional, default is the number of strata (1 if stratification is not used). Only consider setting this option if the stratification groups do not have equal size (the design is not balanced). Option -increment is the stepsize for the total sample size (sum of samples from all strata), i.e. the distance on the x-axis between the points in a total sample size vs. power plot. See section Discussion on strata sample sizes for a more detailed discussion of this option. 
\nextopt
\optdefault{max\_size}{N}
The largest total sample size to plot for, optional, default equal to number of individuals in dataset. 
\nextopt
\optdefault{target\_power}{X}
Default 98. If the whole power curve is generated (option -curve is set), the computations will stop when the computed power exceeds the target power 3 times in a row, even if max\_size is not reached. If option curve is not set, then the program determines the sample size only for the target power. 
\nextopt
\optdefault{table\_full}{filename}
Optional. The name of the table containing iofv for the full model. If this option is used, PsN will skip the estimation of the full model and read the table directly instead. For this option, iotab tables generated using PsN with option -iofv and NONMEM6 will work, as well as .phi-files generated with NONMEM7. Files generated in other ways must follow the following rules: There must be exactly one row per individual, all other lines (headers) must start with a non-number, the columns must be space or tab separated, and iofv must come in the last column. 
\nextopt
\optdefault{table\_reduced}{filename}
Optional. The name of the table containing iofv for the reduced model. If this option is used, PsN will skip the estimation of the reduced model and read the table directly instead. The file must have the format defined above in the -table\_full help text. 
\nextopt
\optdefault{simdata}{filename}
Optional. The name of a previously generated file with simulated data. Cannot be used together with option -simulation\_model. If this option is set, no simulation will be performed by mcmp. Instead the file simdata will be used as the datafile when estimating the full and reduced model. 
\nextopt
\optdefault{table\_strata}{filename}
Optional unless both -table\_full and -table\_reduced is used and stratify\_on is set, then required. Table with stratification column. Must have only one row per individual, i.e. for example generated in NONMEM with FIRSTONLY (see option -stratify\_on), and must have a header with ID and stratification variable name. Option -table\_strata may be used even if neither of -table\_full or -table\_reduced are used, then the stratification column in -table\_strata will be used instead of a table generated from the estimation of the full or reduced model. 
\nextopt
\optdefault{n\_bootstrap}{N}
Optional, default 10000. The number of bootstrapped delta-ofv:s to generate for each total sample size. 
\nextopt
\optdefault{df}{N}
Optional, default 1. The number of degrees of freedom for the chi-square test. Allowed values are 1-20.
PsN will compute the power for significance levels 20\%, 15\%, 10\%, 5\%, 1\% and 0.1\% 
for the number of degrees of freedom. 
\nextopt
\optname{significance\_level}
Optional, default 5 (percent). Permitted values are 20, 15, 10, 5, 1 and 0.1. 
Convergence check will be based on critical ofv for this significance level. 
\nextopt
\optname{critical\_ofv}
Optional, no default. If specified, -critical\_ofv will override the setting of -df. PsN will work with positive values internally (delta\_ofv=reduced-full, check if delta\_ofv $>$ critical\_ofv), but will automatically change the sign if the user gives a negative value. 
\nextopt
\end{optionlist}

\subsection{Some important common PsN options}
For a complete list see common\_options.pdf, 
or psn\_options -h on the commandline.
\begin{optionlist}
\optname{h or -?}
Print the list of available options and exit. 
\nextopt
\optname{help}
With -help all programs will print a longer help message. 
If an option name is given as argument, help will be printed for this option. 
If no option is specified, help text for all options will be printed. 
\nextopt
\optdefault{directory}{'string'}
Default \guidetoolname\_dirN,
where N will start at 1 and
be increased by one each time you run the script. The directory option sets the directory in which PsN 
will run NONMEM and where PsN-generated output files will be stored. 
You do not have to create the directory,  it will be done for you. If you set
-directory to a the name of a directory that already exists, PsN will run in the existing directory.
\nextopt
\optdefault{seed}{'string'}
You can set your own random seed to make PsN runs reproducible.
The random seed is a string, so both -seed=12345 and -seed=JustinBieber are valid.
It is important to know that because of the way the Perl pseudo-random
number generator works, for two similar string seeds the random sequences may be identical. 
This is the case e.g. with the two different seeds 123 and 122. 
Setting the same seed guarantees the same sequence, but setting two slightly different 
seeds does not guarantee two different random sequences, that must be verified.
\nextopt
\optdefault{clean}{'integer'}
Default 1. The clean option can take four different values:  
\begin{description}
\item[0] Nothing is removed 
\item[1] NONMEM binary and intermediate files except INTER are removed, and files specified with option -extra\_files. 
\item[2] model and output files generated by PsN restarts are removed, and data files in the NM\_run directory, and (if option -nmqual is used) the xml-formatted NONMEM output. 
\item[3] All NM\_run directories are completely removed. If the PsN tool has created modelfit\_dir:s inside the main run directory, these  will also be removed. 
\end{description}
\nextopt
\optname{zip}
If this option is set the m1 folder in the run directory will be zipped into the file m1.zip and removed. This will save space and reduce the number of files generated by PsN. PsN can seamlessly handle zipped m1 folders for example when rerunning a command in the same directory using the -directory option.
\nextopt
\optdefault{nm\_version}{'string'}
Default is 'default'. 
If you have more than one NONMEM version installed you can use option
-nm\_version to choose which one to use, as long as it is 
defined in the [nm\_versions] section in psn.conf, see psn\_configuration.pdf for details. 
You can check which versions are defined, without opening psn.conf, using the command
\begin{verbatim}
psn -nm_versions
\end{verbatim}
\nextopt
\optdefault{threads}{'integer'}
Default 5 (if default PsN config file is used). 
Use the threads option to enable parallel execution of multiple models.
This option decides how many models PsN will run at the same time, and it is completely
independent of whether the individual models are run with serial NONMEM or parallel NONMEM.
If you want to run a single model in parallel you must use options -parafile and -nodes.
On a desktop computer it 
is recommended to not set -threads higher the number of CPUs in the system plus one. 
You can specify more threads, 
but it will probably not increase the performance. If you are running on a computer cluster, 
you should consult your 
system administrator to find out how many threads you can specify. 
\nextopt
\optname{version}
Prints the PsN version number of the tool, and then exit. 
\nextopt
\optname{citations}
Print a list of references for this tool. The list will be in BibTeX format.
\nextopt
\end{optionlist}


\subsection{Auto-generated R-plots from PsN}
\newcommand{\rplotsconditions}{The default mcmp template 
requires the R library ggplot2.
If the package is not installed then no pdf will be generated,
see the .Rout file in the main run directory for error messages.}

PsN can automatically generate R plots to visualize results, if a template file is available.
The PsN installation package includes default template files for a number of tools,
but R plots can be generated for any tool if the user provides the template file.
PsN will create a preamble with some run specific information, 
such as the name of the raw\_results file, the parameter labels and the name of
the tool results file,
and then append 
the template file.
If the template file contains an old preamble, then that preamble will be replaced with 
the new one. This means
the user can modify an old PsN-generated R script and then use this script as a new template,
without having to remove the preamble.
When a template file is available for the tool and option \mbox{-rplots} is 0 or positive, 
the R script will be generated and saved in the main
run directory. 
If R is configured in psn.conf or command 'R' is available and option -rplots is positive 
the script will also be run and a number of pdf-format plots be created.

\rplotsconditions

\begin{optionlist}
\optdefault{rplots}{level}
-rplots<0 means R script is not generated\\ 
-rplots=0 (default) means R script is generated but not run\\ 
-rplots=1 means basic plots are generated\\													  
-rplots=2 means basic and extended plots are generated\\													  
\nextopt
\optdefault{template\_file\_rplots}{file}
When the rplots feature is used, the default template file PsN will use is 
\guidetoolname\_default.R. 
The user can choose a different template file
by setting option -template\_file\_rplots to a different file. 
PsN will first look for the file relative to the current working directory, 
and after that in the -template\_directory\_rplots directory.
\nextopt
\optdefault{template\_directory\_rplots}{path}
PsN can look for the rplots template file in a number of places. The priority order is the
following:
\begin{enumerate}
\item template\_directory\_rplots set on command-line 
\item calling directory (where PsN is started)
\item template\_directory\_rplots set in psn.conf 
\item R-scripts subdirectory of the PsN installation directory
\end{enumerate}
\nextopt
\optdefault{subset\_variable\_rplots}{variable name}
Default not set. The user can specify a subset variable to be used with the -rplots feature. 
This variable
will, if set, be set as subset.variable in the preamble,
and can then be used in the plotting code. 
\nextopt
\end{optionlist}


\subsubsection*{Basic plots}
A basic mcmp rplot will be generated in file PsN\_mcmp\_plots.pdf if option -rplots is set >0,
and the general rplots conditions fulfilled, see above.
The basic plot is the mcmp-generated power curve for
the signficance level chosen with option -significance\_level, with a mark at the sample size 
required for 80\% power.
\subsubsection*{Extended plots}
Extended mcmp rplots will be generated in file PsN\_mcmp\_plots.pdf if option -rplots is set >1,
and the general rplots conditions fulfilled, see above.
The first extended plot is the the mcmp-generated power curves for
each of the significance levels, with a mark of the sample size 
required for 80\% power.
The second extended plot is a histogram of the individual full-reduced delta-ofv. 

\section{Output}
The output from each mcmp is mcmp\_results.csv which contains a table with header 
total\_X,power at 20\%,power at 15\%,power at 10\%,power at 5\%,power at 1\%,power at 0.1\%, (headers for N from each stratum). 
One row for each total sample size (total\_X). The table can be plotted in excel. The results are also printed to screen as they are produced, and the user can terminate the run with Ctrl-C if the obtained power is deemed sufficient. The file mcmp\_results.csv will contain the values that were computed before the interruption. If mcmp is rerun the previous result file will be copied and kept.

\section{Known bugs}
If estimation of the full and/or reduced model fails but NONMEM 7 still produces a .phi-file with the initial individual ofv-values, 
then PsN will not detect the failed estimation but continue to bootstrap from the .phi-file. Then the power curves will be completely wrong.

The mcmp requires that -nm\_output=phi or -nm\_output=phi,\emph{more file extensions} is set, either on the command-line or in psn.conf.
The psn.conf that can be auto-generated by the PsN installation script gives the correct settings.

\section{Recovering a crashed/stopped mcmp}
If the simulation finished but none of the estimations finished, then start over in a new run directory but use option -simdata with the dataset from the simulation step. If one or two of the estimations finished, then start in new directory using option -table\_reduced and/or -table\_full and possibly -table\_strata. 

\section{Discussion on strata sample sizes}
The increment and start\_size options may seem complicated, so here is a detailed background to the design of those options in PsN.

We start with some examples to explain a method which is not implemented in PsN. When the design is perfectly balanced, choosing the number of individuals to sample from each stratum in each iteration is trivial. It is more complicated to define an algorithm that works in all cases with any design. Consider the following three cases:

\begin{enumerate}
	\item $N_{total}=400$ where $N_A=200$ and $N_B=200$
	\item $N_{total}=400$ where $N_A=100$ and $N_B=300$
	\item	$N_{total}=424$ where $N_A=233$ and $N_B=191$
\end{enumerate}
It is easy to see that the ideal sampling scheme in case 1 is to take 1 individual from each stratum in the first iteration, 2 from each in the second iteration, and so on. Then the 1:1 balance is perfect in every iteration. The increment, the increase of the total sample size in each iteration, is 2 in this case.

Case 2 is also simple, take 1 individual from stratum A and 3 from stratum B in the first iteration, 2 from stratum A and 6 from stratum B in the second iteration and so on, thus always preserving the 1:3 relation of the group sizes. 

Case 3 is more difficult and it is not obvious what the best strategy is. 

The implicit strategy in for case 1 and 2 is the following algorithm: 

\begin{enumerate}
	\item Find the greatest common divisor D of the strata sizes.
	\item In each iteration, increase the sample size from stratum A with $N_A$/D and from stratum B with $N_B$/D
\end{enumerate}
For case 1 D=200 and $N_A$/D=$N_B$/D=1 and for case 2 D=100, $N_A$/D=1 and $N_B$/D=3. For case 3 it turns out the greatest common divisor is 1, giving a useless strategy since it is not okay to sample 233+191 individuals in the first iteration. Hence finding the greatest common divisor is not a strategy which is suitable to implement in PsN.

PsN must be able to handle all cases, keep as good a balance as possible between the strata regardless of their original sizes, make small enough increases in the sample sizes to give a good power plot and allow the user to affect the sampling strategy as much as possible without making the input options too complicated.

PsN uses the following algorithm:
\begin{enumerate}
	\item Compute the desired total sample size $X_{desired_total}$=start\_size+(i-1)*increment, where i, i=1,2,3... 	is the iteration number, increment has default equal the number of strata but can be set by the 	user, and start\_size has default 3*increment but can be set by the user.
	\item Compute the number of individuals Xi to sample from stratum i using the following formula: 	Xi =round\_to\_nearest\_integer($X_{desired_total}$*$N_i$/$N_{total}$). 
\end{enumerate}
With the formula in step 2 stratum i:s fraction of the total sample size will always be as close as possible to the fraction of stratum i in the original population. The rounding is necessary since the division often has a non-zero remainder, and it is important to note that the actual total sample size $X_{actual_total}$, which is the sum of the individual sample sizes Xi, can differ slightly from $X_{desired_total}$ because of the necessary rounding. If the increment is small it can happen that $X_{actual_total}$ is the same in two consecutive iterations, without there being any error. PsN always reports $X_{actual_total}$ in all output. $X_{desired_total}$ is an internal variable and is never displayed.

If the user has a dataset as case 2 ($N_{total}$=400 where $N_A$=100 and $N_B$=300) the user can set increment to 4 (1+3) which would give a perfect relation between the strata sample sizes in each iteration, since according to PsN:s algorithm stratum A will always get 100/400=1/4 of the samples and stratum B 300/400=3/4 of the samples and $X_{desired_total}$ would always be a multiple of 4. If the user leaves increment to the default = the number of strata, then the results would still be acceptable. In every other iteration the relation would be perfect ( $X_{desired_total}$ is a multiple of 4). In the rest of the iterations the actual relation would deviate slightly, e.g. if $X_{desired_total}$=10 then XA= round(10*100/400)=3 and XB= round(10*300/400)=8 giving $X_{actual_total}$=3+8=11, but the larger the total sample size is the smaller the deviation will be. 

The user can also choose to set start\_size to manipulate the sample sizes, but it is recommended not to set this option to anything other than a multiple of increment.

\section{Technical overview of algorithm}

\begin{enumerate}
\item If NONMEM6 is used, then the iofv option to PsN is set automatically. If NONMEM7 is used, no extra settings are needed.
\item PsN checks that the requirements on the options are fulfilled (see list of options above).
\item Look-up critical ofv if not given on command-line, then make sure sign is +.
\item Unless option simdata is given, or both table\_reduced and table\_full are given, simulate the simulation model with a random number seed in \$SIM which set by PsN.
\item If -reduced\_model is specified, PsN will add a \$TABLE to the first \$PROBLEM with ID $\langle$stratify\_on$\rangle$ FIRSTONLY NOAPPEND ONEHEADER NOPRINT FILE=strata.tab. PsN does not check that it is possible to request $\langle$stratify\_on$\rangle$ in \$TABLE, so it is the responsibility of the user to either have it in \$INPUT or define it in \$PK/\$PRED/\$ERROR.
\item If -reduced\_model is not specified but -full\_model is, then  PsN will add a \$TABLE to the first \$PROBLEM with ID $\langle$stratify\_on$\rangle$ FIRSTONLY NOAPPEND ONEHEADER NOPRINT FILE=strata.tab.
\item In both the reduced and full model, set \$DATA to the simulated data file.
\item Estimate the reduced model unless -table\_reduced is specified.
\item Estimate the full model unless -table\_full is specified.
\item Extract iofv-values from full and reduced iofv-tables, and subtract to create delta-iofv-vector.
\item Stratify delta-iofv-table based on strata.tab/-table\_stratify.
\item Loop over total sample size starting on -start\_size and incrementing with -increment in each step, continuing as long as total sample size does not exceed -max\_size. For each total sample size and each stratum, compute the number of samples to draw from this stratum using the formula
Z=round\begin{math}
(total.sample.size\cdot\frac{n.individuals.in.stratum}{n.individuals.in.total})
\end{math}
. Draw Z samples from the stratum, randomly with replacement. Repeat n\_bootstrap times, sum delta\_iofv for the samples from all strata to generate n\_bootstrap delta\_ofv:s. Compute percentage of delta\_ofv $>$ critical\_ofv=power. In output table, write line with total\_sample\_size, power, sample\_sizes from each stratum. Halt if power exceeds target\_power \% three times in a row.
\end{enumerate}

\bibliography{PsN}{}
\bibliographystyle{plain}

\end{document}
