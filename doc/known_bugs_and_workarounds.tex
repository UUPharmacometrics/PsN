\documentclass[a4paper,12pt]{article}
\title{Known bugs and workarounds\\ \vspace{2 mm} {\large PsN 3.6.2}}
\date{2013-06-03}

\usepackage[utf8]{inputenc}
\usepackage{verbatim}
\usepackage{longtable}
\begin{document}

\maketitle


\section{Introduction}
Work to fix bugs in PsN is ongoing, but often feature addition is prioritized over bug fixing, especially if there is a known workaround for the bug. This document list the most important known bugs, including workarounds when possible.   

\section{Sensitive format for raw results input file in sse, vpc, npc}

It is essential that if any of the column headers contains a comma, e.g. OMEGA(2,1), then every column header must be enclosed in double quotes. If none of the columns headers contains a comma then all or none of the column headers must be enclosed in double quotes. This is done automatically when PsN creates a raw results file but may be changed if the file is saved in e.g. Excel.


\section{Remove leading empty column of csv-format data file}

Bootstrap will crash if the first column of a csv-format datafile is empty.

\section{Path problem on Windows: use Perl build 5.8.8.20}

On some windows platforms a backward slash in the paths is missing and PsN cannot find the modelfile. Robert Kalicki reports that Perl build 5.8.8.20 on Vista works correctly, while 5.8.8.19 does not work.

\section{Warning on Windows}

Windows users may see a warning 

'defined(@array) is deprecated at lib/output\_subs.pm line 841.(Maybe you should just omit the defined()?)' 

every time PsN is run. This message can be ignored.

\section{Ignore with extended\_grid script}
Use IGNORE=@ in the \$DATA record rather than IGNORE=I (where I can stand for ID). Script will crash with IGNORE=I.

\section{Calling old PsN versions on a Windows system with multiple PsN versions installed}
On unix-type systems where multiple PsN versions are installed it is possible to call any version by adding the version number to the name of the PsN-script, e.g. execute-2.3.2 or bootstrap-3.0.0. This does not work on Windows. To call an older installed PsN version, use 

\begin{verbatim}
perl <full_path_to_PsN_executable\<name_of_PsN_script>
\end{verbatim}

for example

\begin{verbatim}
perl C:\Perl\bin\execute-2.3.2
\end{verbatim}

\section{Missing estimates in raw\_results.csv when first \$PROBLEM uses \$MSFI}
When the first \$PROBLEM in the modelfile uses \$MSFI, and \$THETA, \$OMEGA and \$SIGMA are all missing in that \$PROBLEM, then there will be no theta/omega/sigma headers in raw\_results.csv and the parameter estimates will not be printed to the file.

\section{Data values with more than five significant digits in sse}
The simulated datasets used in sse are \$TABLE output from NONMEM, and NONMEM rounds off values when printing tables. In NONMEM6 1013201 is rounded to 1013200 (five significant digits), and if this makes a significant change to the model estimation, for example if the value is a covariate, then the sse results will be wrong. In NONMEM7 it is possible to set the FORMAT option in \$TABLE to make sure no important information is lost. With NONMEM6 the user must make sure the rounding to five significant digits does not harm the results.


\end{document}
