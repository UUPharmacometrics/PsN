\documentclass[12pt]{article}
\usepackage[a4paper,margin=3cm,footskip=2cm]{geometry}
\usepackage{lmodern}
\usepackage[utf8]{inputenc}
\usepackage[T1]{fontenc}
\usepackage{textcomp}
\usepackage{verbatim}
\usepackage{enumitem}
\usepackage{longtable}
\usepackage{alltt}
\usepackage{ifthen}
\usepackage[backend=biber, maxbibnames=99, defernumbers, sorting=none]{biblatex}
\addbibresource{PsN.bib}
% Reduce the size of the underscore
\usepackage{relsize}
\renewcommand{\_}{\textscale{.7}{\textunderscore}}

\input{inputs/version.tex}

\newcommand{\revisiondate}[1]{
\date{#1}
}
\newcommand{\guidetitle}[2]{
\title{#1\\ \vspace{2 mm} {\large PsN \psnversion}}
\date{Revised #2}
}

\newcommand{\references}{
    \printbibliography
}

\newcommand{\doctitle}[2]{
\title{#1}
\date{#2}
}


\newenvironment{optionlist}{
\renewcommand{\arraystretch}{1.1}
\setlength{\leftmargini}{2.5cm}
\begin{description}
%\setlength{\itemsep}{0ex}
}
{\end{description}}

\newcommand{\optname}[1]{\item{{\bfseries\texttt-#1}\newline}}
\newcommand{\optdefault}[2]{\item{{\bfseries\texttt-#1}{\mbox{ = \it #2}}\newline}}
% optconfig is for options that can only be set in a (scm) config file, always without -
\newcommand{\optconfig}[2]{\item{{\bfseries\texttt#1}{\mbox{ = \it #2}}\newline}}

\newcommand{\nextopt}{}


\hyphenation{NONMEM}
\hyphenation{INPUT}

\guidetitle{Known bugs and workarounds}
%review Kajsa 2013-11-18

\begin{document}

\maketitle


\section{Introduction}
Work to fix bugs in PsN is ongoing, but often feature addition is prioritized over bug fixing, especially if there is a known workaround for the bug. This document list the most important known bugs, including workarounds when possible.   

\section{Sensitive format for raw results input file in sse, vpc, npc}

Any column header that contains a comma, e.g. OMEGA(2,1), must be enclosed in double quotes. This is done automatically when PsN creates a raw results file but may be changed if the file is saved in e.g. Excel.

\section{Remove leading empty column of csv-format data file}

Bootstrap will crash if the first column of a csv-format datafile is empty.

\section{Ignore with extended\_grid script}
Use IGNORE=@ in the \$DATA record rather than IGNORE=I (where I can stand for ID). Script will crash with IGNORE=I.

\section{Missing estimates in raw\_results.csv when first \$PROBLEM uses \$MSFI}
When the first \$PROBLEM in the modelfile uses \$MSFI, and \$THETA, \$OMEGA and \$SIGMA are all missing in that \$PROBLEM, then there will be no theta/omega/sigma headers in raw\_results.csv and the parameter estimates will not be printed to the file.

\section{Data values with more than five significant digits in sse}
The simulated datasets used in sse are \$TABLE output from NONMEM, and NONMEM rounds off values when printing tables. In NONMEM 6 1013201 is rounded to 1013200 (five significant digits), and if this makes a significant change to the model estimation, for example if the value is a covariate, then the sse results will be wrong. In NONMEM7 it is possible to set the FORMAT option in \$TABLE to make sure no important information is lost. With NONMEM 6 the user must make sure the rounding to five significant digits does not harm the results.

\section{boot\_scm -stratify\_on is sensitive to header in the data file}
If the option -stratify\_on is used in boot\_scm, then it the the column number/name in the
data file that PsN will use when finding the stratification column \emph{unless} the model has 
an IGNORE/ACCEPT=list statement in \$DATA. If IGNORE/ACCEPT=list is found 
PsN performs a filtering step where the \$INPUT headers are copied
to the filtered data file, and then the column number/name in
\$INPUT will be used to find the stratification column. 

\end{document}
