\documentclass[12pt]{article}
\usepackage[a4paper,margin=3cm,footskip=2cm]{geometry}
\usepackage{lmodern}
\usepackage[utf8]{inputenc}
\usepackage[T1]{fontenc}
\usepackage{textcomp}
\usepackage{verbatim}
\usepackage{enumitem}
\usepackage{longtable}
\usepackage{alltt}
\usepackage{ifthen}
\usepackage[backend=biber, maxbibnames=99, defernumbers, sorting=none]{biblatex}
\addbibresource{PsN.bib}
% Reduce the size of the underscore
\usepackage{relsize}
\renewcommand{\_}{\textscale{.7}{\textunderscore}}

\input{inputs/version.tex}

\newcommand{\revisiondate}[1]{
\date{#1}
}
\newcommand{\guidetitle}[2]{
\title{#1\\ \vspace{2 mm} {\large PsN \psnversion}}
\date{Revised #2}
}

\newcommand{\references}{
    \printbibliography
}

\newcommand{\doctitle}[2]{
\title{#1}
\date{#2}
}


\newenvironment{optionlist}{
\renewcommand{\arraystretch}{1.1}
\setlength{\leftmargini}{2.5cm}
\begin{description}
%\setlength{\itemsep}{0ex}
}
{\end{description}}

\newcommand{\optname}[1]{\item{{\bfseries\texttt-#1}\newline}}
\newcommand{\optdefault}[2]{\item{{\bfseries\texttt-#1}{\mbox{ = \it #2}}\newline}}
% optconfig is for options that can only be set in a (scm) config file, always without -
\newcommand{\optconfig}[2]{\item{{\bfseries\texttt#1}{\mbox{ = \it #2}}\newline}}

\newcommand{\nextopt}{}


\hyphenation{NONMEM}
\hyphenation{INPUT}

\guidetitle{PsN configuration}{2018-03-02}
\usepackage{hyperref}
\begin{document}

\maketitle
\tableofcontents
\newpage

\section{The quick and easy way}
Let the installation script create psn.conf for you. If you are going to run PsN locally on your computer with a standard NONMEM installation, you are done.

If you already have installed PsN and did not let the script create psn.conf, just redo the PsN installation and let the installation create psn.conf automatically this time, and you are done.

\section{Use an already configured psn.conf}
You can copy psn.conf from a previous working PsN installation to the new installation directory. 

If the old psn.conf was from PsN 3 you need to edit the default settings of nmfe\_options, if set in the old psn.conf, because the syntax of this option has changed between PsN 3 and PsN 4.

\section{psn.conf overview}
The configuration file psn.conf is required to make PsN run correctly. In the configuration file the NONMEM installation directory is specified, together with essential version information. 

In psn.conf it is also possible to specify personal default values, see more information below. Comment lines in psn.conf start with a semicolon (;). Throughout the default psn.conf distributed with PsN there are examples of settings for UNIX and Windows. The user can review the settings, add and remove semicolons, and change selected paths.

\subsection{Location of psn.conf}
PsN will look for psn.conf at two different places. Primarily (depending on operative system) in the user's home directory (UNIX) or on the user's desktop (Windows) and then in the PsN installation directory. If psn.conf is found in the home directory (UNIX) or on the user's desktop (Windows) the settings in this psn.conf will override the settings in the system-wide (PsN installation directory) present psn.conf, which is important to remember when trying to sort out a psn.conf-related problem.

\subsection{Sections in psn.conf}
The organizing of settings in psn.conf is important. They must be set in the correct section.\\
A section starts with a new line where square brackets enclose the section name
\verb|[section_name]|. \\
Example:\\
\verb|[nm_versions]|\\
The only exception is the first section which starts at the beginning of the file and has no name.\\
A section ends where next section begins
\verb|[next_section_name]| or at the end of the file.

\subsection{Setting options in psn.conf}
The user can override source code defaults by setting options in the appropriate section in psn.conf. \textbf{NOTE}: Any settings on the command line will override settings made in psn.conf. 

The option names used in psn.conf are the same as the ones that would be used on the command line.The documentation can be found in the document common\_options.pdf. 

Some options do not take any value on the command line, for example -run\_on\_sge, but if set in psn.conf those options must also have a value\\ (1 or 0). If such an option is set in psn.conf, for example
-run\_on\_sge=1, the option can be disabled on the command line by using -no, for example 
-no-run\_on\_sge. \textbf{NOTE}: The -no prefix cannot be used in psn.conf, instead set the option to =0.
\subsection {Syntax in psn.conf}
The syntax in psn.conf is as follows: Options values to be used for all tools must be set in the section \verb|[default_options]|. 
On each row in the section the option name (without leading minus-sign) comes first. Then there is an equal sign and finally the value. Spaces added around the equal sign are optional. Options which on the command line can be set just by giving the option name, e.g. abort\_on\_fail, must in psn.conf be given the value 1.\\ Examples:
\begin{verbatim}
abort_on_fail = 1
nmfe = 1
\end{verbatim}
Options with values:
\begin{verbatim}
threads = 5
sge_prepend_flags = -V 
\end{verbatim}
Options for specific tools, e.g. llp, can be set in the sections\\
\verb|[default_<scriptname>_options]|, for example\\
 \verb|[default_llp_options]|.

\noindent New sections may be added, example:\\
\verb|[default_sse_options]|

\noindent Default settings made in a tool-specific section will have precedence over the settings in \verb|[default_options]|.

\subsection{-nm\_version dependent default settings}
It is possible to set different default settings for different NONMEM versions. This functionality is intended for advanced users only. Each NONMEM version specified in psn.conf has a name (see section 'Essential NONMEM information') that can be used on the command line to invoke this version, e.g. -nm\_version=nm72.\\
Example with the PsN tool vpc:\\
When -nm\_version=nm72 is set on the vpc command line, PsN will look for sections \verb|[default_options_nm72]| 
and \verb|[default_vpc_options_nm72]| and read options from there in addition to from the regular \verb|[default_options]| and \verb|[default_vpc_options]|. 

The option -nm\_version \emph{must be set on the command line} to invoke this feature. NONMEM-version specific default settings will \emph{not} be used if option nm\_version is only set in psn.conf.
\subsubsection*{Precedence order}
PsN will look for option settings in the following precedence order;
\begin{enumerate}
\item Settings made on the command line have precedence over settings in
\item \verb|[default_scriptname_options_nmversion]| (if present)\\
which have precedence over settings in
\item \verb|[default_scriptname_options]| (if present)\\
which have precedence over settings in
\item \verb|[default_options_nmversion]| (if present)\\
which have precedence over settings in
\item \verb|[default_options]|(if present). 
\end{enumerate}
\subsubsection*{UNIX(Linux)/Windows psn.conf format are different}
\textbf{NOTE}: If running PsN on UNIX/Linux but editing psn.conf on Windows, make sure to save psn.conf in UNIX format or to convert after editing.
\newpage

\section{Essential NONMEM information}
It is necessary to specify the NONMEM installation information in psn.conf. This is done in the section
\begin{verbatim} 
[nm_versions]
\end{verbatim}
in psn.conf. The format is 
\begin{verbatim}
name=installation_directory,version_number
\end{verbatim}
The NONMEM version number should be major.minor, as in 7.2 for example.
The major version number and the subversion number should be entered with a dot in between, no spaces. \\
Examples:\\
UNIX:
\begin{verbatim}
[nm_versions]
;with NONMEM 7.2 
default=/opt/NONMEM/nm73,7.3
7_1=/opt/NONMEM/nm71,7.1
7_2=/opt/NONMEM/nm72,7.2
\end{verbatim}
Windows:
\begin{verbatim}
;with NONMEM 7.3
default=C:/NONMEM/nm74,7.4
nm73=C:/NONMEM/nm73,7.3
nm72=C:/NONMEM/nm72,7.2
\end{verbatim}
By default PsN will use the NONMEM installation identified by the name 'default'. To use other installations with PsN, use the "-nm\_version" command line option, for example -nm\_version=vi\_big.

It is optional to add a comma and a description of the NONMEM installation after the version number. Example:
\begin{verbatim}
[nm_versions]
;with NONMEM 7.2 
7_1=/opt/NONMEM/nm71,7.1,NM 7.1 using gfortran 4.4.7
7_2=/opt/NONMEM/nm72,7.2,NM 7.2 using gfortran 5.1.1
\end{verbatim}

To check which versions are defined in psn.conf without opening the file, use the command psn -nm\_versions.
This command will also show the (if any) optional description text after the version number.

If the user wants to invoke nmfe using a wrapper, or use a modifed nmfe script with a different name, the name of the alternative script including the full path can be set under [nm\_versions] instead of in the installation directory. PsN will check that the file exists and is executable. 
Using a wrapper is useful e.g. when environment variable changes are needed prior to nmfe execution.

\subsection{NONMEM information for portable PsN}
This functionality is intended for advanced users only. The
\verb|[nm_versions]|
section in psn.conf can list the name of the nmfe executables instead of the full path to the NONMEM installation directories \emph{if at runtime} the nmfe executable, chosen by default or with option -nm\_version, is in the user \$PATH. Example:
\begin{verbatim}
[nm_versions]
default=nmfe73,7.3
nm72=nmfe72,7.2
nm712=nmfe7,7.1
\end{verbatim}
PsN will exit with an error message if the executable chosen at runtime is not in the user \$PATH.

The names of the executables can be given without the extension on Windows \emph{if} the extension is either '.bat' or '.exe'. 

The above form of \verb|[nm_versions]| must be manually created in psn.conf. It cannot be automatically generated by the setup script.

\section{Running nmfe from within PsN}

PsN will only run NONMEM via the nmfe script or an alternative script set by the user. 
The default setting is nmfe, where PsN will look for nmfeX, where X stands for the NONMEM version number specified in the nm\_versions section (X is 72 if NONMEM 7.2 is used), in first the /run then the /util and last the /. subdirectory of the installation directory specified in psn.conf, and call the first instance found. Note that it is possible to use nmfeX scripts located in any directory specified in the [nm\_versions] section, not just a standard NONMEM installation directory, and that a wrapper can also be used, see section Essential NONMEM information.

\section{Compiler configuration}

You cannot set any compiler instructions in PsN 4.
\section{Options set before the sections in psn.conf}
There are some options that need to be set at the beginning of psn.conf, before the first \verb|[section_name]| header. A few of these options are covered in the following paragraphs, \textbf{job submission delays}, \textbf{polling interval} and \textbf{output style}.\\
Examples:
\begin{verbatim}
min_fork_delay=100000
max_fork_delay=1
job_polling_interval=2
output_style=MATLAB

[section_name]
\end{verbatim}


\subsection{Job submission delays}
When more than one model will be run in parallel (threads > 1 and more than one model to run), it is possible to
make PsN pause between submissions of new models. The relevant configuration variables are
\verb|min_fork_delay|, the maximum delay in seconds, and \verb|max_fork_delay|, the step-length in micro-seconds between each check if the output from the previous model in the list has appeared.
If the output from the previous run is visible the pause ends.\\
Example:
\begin{verbatim}
min_fork_delay=100000
max_fork_delay=1
\end{verbatim}
which will give at most 1 second pause in steps of 0.1 seconds. Of course, if \verb|min_fork_delay| is set larger than
$10^6\cdot$\verb|max_fork_delay| the delay will be longer than \verb|max_fork_delay| seconds. This method makes
the delay independent of the output generation from other models.

\subsection{Polling interval}
By default, PsN will check for finished jobs every 1 second. This can be changed by setting \verb|job_polling_interval| in psn.conf. However, it cannot be set to 0.
Example:
\begin{verbatim}
job_polling_interval=2
\end{verbatim}
which will make PsN check for finished jobs every 2 seconds.

\subsection{Output style}
There is a possibility to configure the format of the raw results files via the \verb|output_style| configuration option. The alternatives are SPLUS, MATLAB or EXCEL. The setting will control how missing data and the header row is handled in the raw results table. Missing data will be added as 
\begin{itemize}
\item NA for SPLUS
\item NaN for MATLAB 
and 
\item empty for EXCEL 
\end{itemize}
The header row will not be added for MATLAB but for EXCEL and SPLUS.

\end{document}

