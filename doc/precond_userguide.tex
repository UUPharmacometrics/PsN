\documentclass[12pt]{article}
\usepackage[a4paper,margin=3cm,footskip=2cm]{geometry}
\usepackage{lmodern}
\usepackage[utf8]{inputenc}
\usepackage[T1]{fontenc}
\usepackage{textcomp}
\usepackage{verbatim}
\usepackage{enumitem}
\usepackage{longtable}
\usepackage{alltt}
\usepackage{ifthen}
\usepackage[backend=biber, maxbibnames=99, defernumbers, sorting=none]{biblatex}
\addbibresource{PsN.bib}
% Reduce the size of the underscore
\usepackage{relsize}
\renewcommand{\_}{\textscale{.7}{\textunderscore}}

\input{inputs/version.tex}

\newcommand{\revisiondate}[1]{
\date{#1}
}
\newcommand{\guidetitle}[2]{
\title{#1\\ \vspace{2 mm} {\large PsN \psnversion}}
\date{Revised #2}
}

\newcommand{\references}{
    \printbibliography
}

\newcommand{\doctitle}[2]{
\title{#1}
\date{#2}
}


\newenvironment{optionlist}{
\renewcommand{\arraystretch}{1.1}
\setlength{\leftmargini}{2.5cm}
\begin{description}
%\setlength{\itemsep}{0ex}
}
{\end{description}}

\newcommand{\optname}[1]{\item{{\bfseries\texttt-#1}\newline}}
\newcommand{\optdefault}[2]{\item{{\bfseries\texttt-#1}{\mbox{ = \it #2}}\newline}}
% optconfig is for options that can only be set in a (scm) config file, always without -
\newcommand{\optconfig}[2]{\item{{\bfseries\texttt#1}{\mbox{ = \it #2}}\newline}}

\newcommand{\nextopt}{}


\hyphenation{NONMEM}
\hyphenation{INPUT}

\guidetitle{PRECOND user guide}

\begin{document}

\maketitle


\section{Introduction}
The precond script creates a reparametrized modelfile given a matrix to use for preconditioning.

Examples
\begin{verbatim}
precond run1.mod -pre=pmatrix.csv
precond run2.mod -pre=psn.cov -cholesky 
\end{verbatim}

\section{Input and options}

\subsection{Required input}
Required argument is a model file and a preconditioning matrix.

\begin{optionlist}

\optdefault{pre}{psn.cov}
The name of the matrix used for preconditioning. It can be either a comma separated file without header or a NONMEM output covariance matrix.
\nextopt
\end{optionlist}

\subsection{Optional input}

\begin{optionlist}
\optname{cholesky}
Use cholesky decomposition of the preconditioning matrix instead of LU decomposition.
\optname{nodec}
Turn off decomposition of preconditioning matrix.
\optdefault{output\_model}{run1\_repara.mod}
Set name of output model. Setting this option will only create the reparametrized model without running it.
\optdefault{cov}{result.cov}
Only perform a conversion of a reparametrized covariance matrix. If this option is set no model will be run.
\end{optionlist}

\section{Output}

If the option -output\_model is set the script will create the reparametrized model file without running it.
If the option -cov is set the script will convert back a reparametrized covariance matrix.

If none of -output\_model or -cov are used the model will be reparametrized, run and the resulting covariance matrix will be
backconverted and stored as result.cov.


\section{Preconditioning matrix}

The preconditioning matrix can be either a csv file or a NONMEM .cov file. It will be either truncated or padded to the number of thetas in the model.
When padded all new rows will be all zero except for the diagonal element which will be set to one. If any rows in the preconditioning matrix is found
to be all zeros a one will be placed as the diagonal element.

\section{Description}





\end{document}
