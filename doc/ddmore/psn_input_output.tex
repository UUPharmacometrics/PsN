\documentclass[12pt]{article}
\usepackage[a4paper,margin=3cm,footskip=2cm]{geometry}
\usepackage{lmodern}
\usepackage[utf8]{inputenc}
\usepackage[T1]{fontenc}
\usepackage{textcomp}
\usepackage{verbatim}
\usepackage{enumitem}
\usepackage{longtable}
\usepackage{alltt}
\usepackage{ifthen}
\usepackage[backend=biber, maxbibnames=99, defernumbers, sorting=none]{biblatex}
\addbibresource{PsN.bib}
% Reduce the size of the underscore
\usepackage{relsize}
\renewcommand{\_}{\textscale{.7}{\textunderscore}}

\input{inputs/version.tex}

\newcommand{\revisiondate}[1]{
\date{#1}
}
\newcommand{\guidetitle}[2]{
\title{#1\\ \vspace{2 mm} {\large PsN \psnversion}}
\date{Revised #2}
}

\newcommand{\references}{
    \printbibliography
}

\newcommand{\doctitle}[2]{
\title{#1}
\date{#2}
}


\newenvironment{optionlist}{
\renewcommand{\arraystretch}{1.1}
\setlength{\leftmargini}{2.5cm}
\begin{description}
%\setlength{\itemsep}{0ex}
}
{\end{description}}

\newcommand{\optname}[1]{\item{{\bfseries\texttt-#1}\newline}}
\newcommand{\optdefault}[2]{\item{{\bfseries\texttt-#1}{\mbox{ = \it #2}}\newline}}
% optconfig is for options that can only be set in a (scm) config file, always without -
\newcommand{\optconfig}[2]{\item{{\bfseries\texttt#1}{\mbox{ = \it #2}}\newline}}

\newcommand{\nextopt}{}


\hyphenation{NONMEM}
\hyphenation{INPUT}

\guidetitle{PsN input and output}

\begin{document}

\maketitle
\section{Common input and output}
\subsection{Input}
All PsN programs require at least one NONMEM control stream file as input, 
and the data file set in \$DATA must be present.
\subsection{Output}
\begin{description}
\item[raw\_results]
The raw\_results file is common output from PsN programs. It is a structured
comma-separated file containing a table with one header row. Following the header are one result
line per sub-problem per \$PROBLEM per model that was run.
The columns include model, problem and subproblem number, a 
set of booleans depending on the settings in the model
(e.g. estimation\_step\_run), the ofv, parameter estimates, standard errors,
etc. The number of columns and the headers depend on the structure of the model,
comments in the model file, the PsN program generating the file and the options
to the particular run.
The name of the raw\_results file always starts with raw\_results but the end depends on the 
PsN program and the name of the model file.
\item[<program>\_results.csv]
Most PsN programs produce a <program>\_results.csv file, for example bootstrap\_results.csv. 
It is a human readable comma separated file intended for viewing in e.g. Excel. 
It contains run information and 
result summaries. There are multiple sections spanning varying numbers of rows and columns.
\end{description}

\section{Output files per PsN program}
\subsection{bootstrap}
\begin{enumerate}
\item raw\_results 
\item bootstrap\_results.csv 
\end{enumerate}
\subsection{cdd}
\begin{enumerate}
\item raw\_results 
\item cdd\_results.csv 
\end{enumerate}
\subsection{execute}
\begin{enumerate}
\item raw\_results
\item A set of standard NONMEM output files
\end{enumerate}
\subsection{llp}
\begin{enumerate}
\item raw\_results 
\item llp\_results.csv 
\end{enumerate}
\subsection{randtest}
\begin{enumerate}
\item raw\_results 
\end{enumerate}
\subsection{scm}
\begin{enumerate}
\item raw\_results
\item A human readable log file
\item A NONMEM control stream with the final model
\end{enumerate}
\subsection{sse}
\begin{enumerate}
\item raw\_results 
\item sse\_results.csv 
\end{enumerate}
\subsection{vpc}
\begin{enumerate}
\item vpctab, which is used as input to Xpose when results are visualized
\item vpc\_results.csv 
\end{enumerate}

\end{document}
