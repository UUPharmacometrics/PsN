\documentclass[12pt]{article}
\usepackage[a4paper,margin=3cm,footskip=2cm]{geometry}
\usepackage{lmodern}
\usepackage[utf8]{inputenc}
\usepackage[T1]{fontenc}
\usepackage{textcomp}
\usepackage{verbatim}
\usepackage{enumitem}
\usepackage{longtable}
\usepackage{alltt}
\usepackage{ifthen}
\usepackage[backend=biber, maxbibnames=99, defernumbers, sorting=none]{biblatex}
\addbibresource{PsN.bib}
% Reduce the size of the underscore
\usepackage{relsize}
\renewcommand{\_}{\textscale{.7}{\textunderscore}}

\input{inputs/version.tex}

\newcommand{\revisiondate}[1]{
\date{#1}
}
\newcommand{\guidetitle}[2]{
\title{#1\\ \vspace{2 mm} {\large PsN \psnversion}}
\date{Revised #2}
}

\newcommand{\references}{
    \printbibliography
}

\newcommand{\doctitle}[2]{
\title{#1}
\date{#2}
}


\newenvironment{optionlist}{
\renewcommand{\arraystretch}{1.1}
\setlength{\leftmargini}{2.5cm}
\begin{description}
%\setlength{\itemsep}{0ex}
}
{\end{description}}

\newcommand{\optname}[1]{\item{{\bfseries\texttt-#1}\newline}}
\newcommand{\optdefault}[2]{\item{{\bfseries\texttt-#1}{\mbox{ = \it #2}}\newline}}
% optconfig is for options that can only be set in a (scm) config file, always without -
\newcommand{\optconfig}[2]{\item{{\bfseries\texttt#1}{\mbox{ = \it #2}}\newline}}

\newcommand{\nextopt}{}


\hyphenation{NONMEM}
\hyphenation{INPUT}

\guidetitle{PVAR user guide}{2017-12-18}

\newcommand{\guidetoolname}{pvar}

\begin{document}

\maketitle


\section{Introduction}
The pvar (parametric variability) tool calculates how much of parameter variabilty that a model explains \cite{Hennig}. For every parameter listed on the command line PsN will calculate the total variance, the explained variance and the unexplained variance. The resulting table can be found in the result.csv. The input models to pvar must have final parameter estimates.\\
Examples:
\begin{verbatim}
pvar scmlog1.txt -parameters=CL,V
\end{verbatim}
Check parametric variability of models generated in an scm run.
\begin{verbatim}
pvar -models run1.mod run2.mod -parameters=CL,V
\end{verbatim}
Use specific models.
\section{Input and options}

\subsection{Required input}
Required arguments are either an scm log file or a list of model files and a list of parameters. If an scm log file is to be used the scm run cannot have been made with -clean=3 otherwise files needed for pvar will be deleted.

\begin{optionlist}
\optdefault{parameters}{CL,V,...}
A mandatory comma separated list of parameter to investigate.
\nextopt
\end{optionlist}

\subsection{Optional input}

\begin{optionlist}
\optname{models}
If this option is present on the command line a list of model files can be passed as arguments instead of an scm logfile.\\ Example:
\begin{verbatim}
pvar -models run1.mod run2.mod -parameters=CL,V
\end{verbatim}
\optdefault{samples}{N}
Default is 100. Number of simulated datasets to generate. 
\nextopt
\end{optionlist}

\section{Run pvar from a scm}
The pvar tool can be run for all the intermediate models of a scm run. For this to work the scm must have been run with a clean option less than or equal to two (i.e. -clean=0, -clean=1 or -clean=2). The scm log file of the scm run must be specified as an argument to the pvar run. From the log file pvar will find which models to use. Note that they will be numbered in the results file in the order they appeared in the log file. The base model of the scm will always be run as the first model (number 0). pvar can not be run on a linearized scm. 

\section{Result}

The result.csv is formatted as follows:

The first row is the header explaining the columns.
\begin{verbatim}
Type,Model number,Model name,Run number,CL,V,OFV
\end{verbatim}
\begin{itemize}
	\item The first column is the type of variability (EPV for explained, UPV for unexplained and PV for total). 
	\item The second column is the number of the model starting with zero. They will come in the order of the scmlog or in the order specified on the command line when using the models option.
	\item The third column is the name of the model. 
	\item The following columns until the last column are the variances for the parameters. 
	\item The last column is the OFV for this model.
\end{itemize}

Example of result.csv:
\begin{verbatim}
Type,Model number,Model name,Run number,CL,V,OFV
EPV,0,run101.mod,101,1.64942442067797e-06,0.0620681720338984,527.08
UPV,0,run101.mod,101,7.1337349188431e-06,0.538948190857009,527.08
PV,0,run101.mod,101,8.78315933952107e-06,0.601016362890907,527.08
EPV,1,run103.mod,103,1.5230948760263e-05,0.0468773796610169,591.443
UPV,1,run103.mod,103,1.75795164007126e-06,0.387259333476643,591.443
PV,1,run103.mod,103,1.69889004003343e-05,0.43413671313766,591.443
\end{verbatim}

\subsection{Auto-generated R-plots from PsN}
\newcommand{\rplotsconditions}{The default pvar template requires no extra R libraries.}

PsN can automatically generate R plots to visualize results, if a template file is available.
The PsN installation package includes default template files for a number of tools,
but R plots can be generated for any tool if the user provides the template file.
PsN will create a preamble with some run specific information, 
such as the name of the raw\_results file, the parameter labels and the name of
the tool results file,
and then append 
the template file.
If the template file contains an old preamble, then that preamble will be replaced with 
the new one. This means
the user can modify an old PsN-generated R script and then use this script as a new template,
without having to remove the preamble.
When a template file is available for the tool and option \mbox{-rplots} is 0 or positive, 
the R script will be generated and saved in the main
run directory. 
If R is configured in psn.conf or command 'R' is available and option -rplots is positive 
the script will also be run and a number of pdf-format plots be created.

\rplotsconditions

\begin{optionlist}
\optdefault{rplots}{level}
-rplots<0 means R script is not generated\\ 
-rplots=0 (default) means R script is generated but not run\\ 
-rplots=1 means basic plots are generated\\													  
-rplots=2 means basic and extended plots are generated\\													  
\nextopt
\optdefault{template\_file\_rplots}{file}
When the rplots feature is used, the default template file PsN will use is 
\guidetoolname\_default.R. 
The user can choose a different template file
by setting option -template\_file\_rplots to a different file. 
PsN will first look for the file relative to the current working directory, 
and after that in the -template\_directory\_rplots directory.
\nextopt
\optdefault{template\_directory\_rplots}{path}
PsN can look for the rplots template file in a number of places. The priority order is the
following:
\begin{enumerate}
\item template\_directory\_rplots set on command-line 
\item calling directory (where PsN is started)
\item template\_directory\_rplots set in psn.conf 
\item R-scripts subdirectory of the PsN installation directory
\end{enumerate}
\nextopt
\optdefault{subset\_variable\_rplots}{variable name}
Default not set. The user can specify a subset variable to be used with the -rplots feature. 
This variable
will, if set, be set as subset.variable in the preamble,
and can then be used in the plotting code. 
\nextopt
\end{optionlist}


\subsubsection*{Basic plot}
Basic rplots will be generated if option -rplots is set > 0, and the general rplots conditions fulfilled. The generated set of plots are barplots for each parameter that shows the explained and unexplained variabilty for each model.


\section{Algorithm}

Simulate each model in one original version to get the total variability and one with all omegas set to zero to get the explained variability.

\begin{enumerate}
	\item Input: List of models.
	\item For each model:
	\begin{enumerate}
		\item Create two new models. One for EPV and one for PV.
		\item If model has been run update the initial values from results.
		\item Remove \$EST.
		\item If model already contain \$SIMULATION give a warning but keep \$SIMULATION.
		\item Add \$SIMULATION (<seed>) NSUBS=<samples> ONLYSIM.
		\item Add \$TABLE NOPRINT NOAPPEND FIRSTONLY ONEHEADER ID FILE=<EPV|PV><n>.tab <par1> <par2> ...
		\item Set \$OMEGA 0 FIX for the EPV model.
	\end{enumerate}
	\item Run models.
	\item Loop through table-files to calculate the variances.
\end{enumerate}

\references

\end{document}
