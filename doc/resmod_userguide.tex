\documentclass[12pt]{article}
\usepackage[a4paper,margin=3cm,footskip=2cm]{geometry}
\usepackage{lmodern}
\usepackage[utf8]{inputenc}
\usepackage[T1]{fontenc}
\usepackage{textcomp}
\usepackage{verbatim}
\usepackage{enumitem}
\usepackage{longtable}
\usepackage{alltt}
\usepackage{ifthen}
\usepackage[backend=biber, maxbibnames=99, defernumbers, sorting=none]{biblatex}
\addbibresource{PsN.bib}
% Reduce the size of the underscore
\usepackage{relsize}
\renewcommand{\_}{\textscale{.7}{\textunderscore}}

\input{inputs/version.tex}

\newcommand{\revisiondate}[1]{
\date{#1}
}
\newcommand{\guidetitle}[2]{
\title{#1\\ \vspace{2 mm} {\large PsN \psnversion}}
\date{Revised #2}
}

\newcommand{\references}{
    \printbibliography
}

\newcommand{\doctitle}[2]{
\title{#1}
\date{#2}
}


\newenvironment{optionlist}{
\renewcommand{\arraystretch}{1.1}
\setlength{\leftmargini}{2.5cm}
\begin{description}
%\setlength{\itemsep}{0ex}
}
{\end{description}}

\newcommand{\optname}[1]{\item{{\bfseries\texttt-#1}\newline}}
\newcommand{\optdefault}[2]{\item{{\bfseries\texttt-#1}{\mbox{ = \it #2}}\newline}}
% optconfig is for options that can only be set in a (scm) config file, always without -
\newcommand{\optconfig}[2]{\item{{\bfseries\texttt#1}{\mbox{ = \it #2}}\newline}}

\newcommand{\nextopt}{}


\hyphenation{NONMEM}
\hyphenation{INPUT}

\guidetitle{RESMOD user guide}{2017-01-11}

\begin{document}

\maketitle
\newcommand{\guidetoolname}{resmod}


\section{Overview}
The residual modelling tool performs modelling of the conditional weighted residual output of a model run. Different types of models are run and the OFV of each will be presented.

Example
\begin{verbatim}
resmod run1.mod
\end{verbatim}

\section{Input and options}

\subsection{Required input}
A model file is required on the command-line.

\subsection{Optional input}

\begin{optionlist}
\optdefault{idv}{TIME}
Name of the independent variable. Set to TIME by default.
\nextopt
\end{optionlist}

\section{Output}
The OFVs of all models are summarized in the results.csv file.

\section{Residual models}

\subsection{Base model}
Named \verb|base| in results.csv.

\begin{verbatim}
Y = THETA(1) + ETA(1) + ERR(1)
\end{verbatim}

\subsection{Laplace}
Named \verb|laplace| in results.csv.

\begin{verbatim}
Y = THETA(1) + ETA(1) + ERR(1)
\end{verbatim}

Using the Laplace estimation method.

\subsection{Laplace 2LL}
Named \verb|laplace_2ll_100df| and \verb|laplace_2ll_dfest| in results.csv. The first variant have 100 degrees of freedom and the second variant estimates the degrees of freedom.

\begin{verbatim}
IPRED = THETA(1) + ETA(1)
W = THETA(2)
DF = THETA(3) ; degrees of freedom of Student distribution
SIG1 = W ; scaling factor for standard deviation of RUV
IWRES = (DV - IPRED) / SIG1
; Nemesapproximation of gamma funtion(2007) for
; first factor of t-distrib(gamma((DF+1)/2))
PHI = (DF + 1) / 2
INN = PHI + 1 / (12 * PHI - 1 / (10 * PHI))
GAMMA = SQRT(2 * 3.14159265 / PHI) * (INN / EXP(1)) ** PHI
; Nemesapproximation of gamma funtion(2007) for
; second factor of t-distrib(gamma(DF/2))
PHI2 = DF / 2
INN2 = PHI2 + 1 / (12 * PHI2 - 1 / (10 * PHI2))
GAMMA2 = SQRT(2*3.14159265/PHI2)*(INN2/EXP(1))**PHI2
; coefficient of PDF of t-distribution
COEFF=GAMMA/(GAMMA2*SQRT(DF*3.14159265))/SIG1
BASE=1+IWRES*IWRES/DF ; base of PDF of t-distribution
POW=-(DF+1)/2 ; power of PDF of t-distribution
L=COEFF*BASE**POW ; PDF oft-distribution
Y=-2*LOG(L)
\end{verbatim}

\subsection{Eta on epsilon}
Named \verb|eta_on_epsilon| in results.csv.

\begin{verbatim}
Y = THETA(1) + ETA(1) + ERR(1) * EXP(ETA(2))
\end{verbatim}

\subsection{Combined time varying}
Named \verb|time_varying_all| in results.csv. The <idv> below will be replaced with the name of the independent variable. The <q1>, <q2> and <q3> below will be replaced with the quartiles of the independent variable
for all observations combined. I.e. from the data set remove all records for which CWRES = 0 then calculate the quartiles of the idv column.
\begin{verbatim}
Y = THETA(1) + ETA(1) + ERR(4)
IF (<idv>.LT.<q1>) THEN
    Y = THETA(1) + ETA(1) + ERR(1)
            'END IF
IF (<idv>.GE.<q1> .AND. <idv>.LT.<q2>) THEN
    Y = THETA(1) + ETA(1) + ERR(2)
END IF
IF (<idv>.GE.<q2> .AND. <idv>.LT.<q3>) THEN
    Y = THETA(1) + ETA(1) + ERR(3)
END IF
\end{verbatim}

\subsection{Time varying first, second or third quartile cutoffs}
Named \verb|time_varying_q1|, \verb|time_varying_q2| and \verb|time_varying_q3| in results.csv. The <idv> and <qx> will be replaced as in the combined time varying model above.
\begin{verbatim}
Y = THETA(1) + ETA(1) + ERR(2)
IF (<idv>.LT.<qx>) THEN
    Y = THETA(1) + ETA(1) + ERR(1)
END IF
\end{verbatim}

\subsection{AR1}
Named \verb|AR1| in results.csv.
\begin{verbatim}
"FIRST
" USE SIZES, ONLY: NO
" USE NMPRD_REAL, ONLY: C=>CORRL2
" REAL (KIND=DPSIZE) :: T(NO)
" INTEGER (KIND=ISIZE) :: I,J,L
"MAIN
"C If new ind, initialize loop
" IF (NEWIND.NE.2) THEN
"  I=0
"  L=1
"  OID=ID
" END IF
"C Only if first in L2 set and if observation
"C  IF (MDV.EQ.0) THEN
"  I=I+1
"  T(I)=TIME
"  IF (OID.EQ.ID) L=I
"
"  DO J=1,I
"      C(J,1)=EXP((-0.6931/THETA(2))*(TIME-T(J)))
"  ENDDO
Y = THETA(1) + ETA(1) + EPS(1)
\end{verbatim}

\subsection{Power IPRED}
Named \verb|power_ipred| in results.csv.
\begin{verbatim}
'Y = THETA(1) + ETA(1) + ERR(1)*(IPRED)**THETA(2)
\end{verbatim}

\subsection{DTBS base model}
Named \verb|dtbs_base| in results.csv.
\begin{verbatim}
IPRED = THETA(1) + ETA(1)
IF(IPRED.LT.0) IPRED=0.0001
W = THETA(2)
Y = IPRED + ERR(1)*W
IF(ICALL.EQ.4) Y=EXP(DV)
\end{verbatim}

\subsection{DTBS}
Named \verb|dtbs| in results.csv.

\begin{verbatim}
$SUBROUTINE CONTR=contr.txt CCONTR=ccontra.txt
$PRED
IPRT = THETA(1)+ETA(1)
LAMBDA = THETA(3)
ZETA   = THETA(4)
IF(IPRT.LT.0) IPRT=0.0001
W = THETA(2)*IPRED**ZETA
IPRTR = IPRT
IF (LAMBDA .NE. 0 .AND. IPRT .NE.0) THEN
	IPRTR = (IPRT**LAMBDA-1)/LAMBDA
ENDIF
IF (LAMBDA .EQ. 0 .AND. IPRT .NE.0) THEN
    IPRTR = LOG(IPRT)
ENDIF
IF (LAMBDA .NE. 0 .AND. IPRT .EQ.0) THEN
	IPRTR = -1/LAMBDA
ENDIF
IF (LAMBDA .EQ. 0 .AND. IPRT .EQ.0) THEN
	IPRTR = -1000000000
ENDIF
IPRT = IPRTR
Y = IPRT + ERR(1)*W
IF(ICALL.EQ.4) Y=EXP(DV)
\end{verbatim}

\end{document}
