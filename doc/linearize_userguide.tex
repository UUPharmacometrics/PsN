\documentclass[12pt]{article}
\usepackage[a4paper,margin=3cm,footskip=2cm]{geometry}
\usepackage{lmodern}
\usepackage[utf8]{inputenc}
\usepackage[T1]{fontenc}
\usepackage{textcomp}
\usepackage{verbatim}
\usepackage{enumitem}
\usepackage{longtable}
\usepackage{alltt}
\usepackage{ifthen}
\usepackage[backend=biber, maxbibnames=99, defernumbers, sorting=none]{biblatex}
\addbibresource{PsN.bib}
% Reduce the size of the underscore
\usepackage{relsize}
\renewcommand{\_}{\textscale{.7}{\textunderscore}}

\input{inputs/version.tex}

\newcommand{\revisiondate}[1]{
\date{#1}
}
\newcommand{\guidetitle}[2]{
\title{#1\\ \vspace{2 mm} {\large PsN \psnversion}}
\date{Revised #2}
}

\newcommand{\references}{
    \printbibliography
}

\newcommand{\doctitle}[2]{
\title{#1}
\date{#2}
}


\newenvironment{optionlist}{
\renewcommand{\arraystretch}{1.1}
\setlength{\leftmargini}{2.5cm}
\begin{description}
%\setlength{\itemsep}{0ex}
}
{\end{description}}

\newcommand{\optname}[1]{\item{{\bfseries\texttt-#1}\newline}}
\newcommand{\optdefault}[2]{\item{{\bfseries\texttt-#1}{\mbox{ = \it #2}}\newline}}
% optconfig is for options that can only be set in a (scm) config file, always without -
\newcommand{\optconfig}[2]{\item{{\bfseries\texttt#1}{\mbox{ = \it #2}}\newline}}

\newcommand{\nextopt}{}


\hyphenation{NONMEM}
\hyphenation{INPUT}

\guidetitle{LINEARIZE user guide}{2017-11-17}


\begin{document}

\maketitle
\newcommand{\guidetoolname}{linearize}


\section{Introduction}
The linearize tool allows you to automatically create a linearized version of a model and obtain the dataset including
individual predictions and derivatives necessary for further estimation of extensions implemented in the linearized model.\\ 
Example:
\begin{verbatim}
linearize run10.mod
\end{verbatim}

The linearization was developed with the aim to facilitate the development of nonlinear mixed effects models by establishing a diagnostic method for evaluation of stochastic model components. A full description of the methodology and its performance 
is presented in \cite{Svensson}. The online supplementary material includes a comprehensive code example. 
The methodology utilizes first order Taylor expansions and substantially shortens run times. Examples of extensions that can be evaluated with the linearized model are addition of inter individual or inter occasion variability parameters, correlation structures and more complex residual error models.

Before proceeding with implementation and evaluation of extensions, it is important to check that the OFV value of the nonlinear and linearized version of the base model agrees (printed in the command window and in the linlog.txt file generated in the linearization folder). If the OFV-values differ more than a few points, this can depend on the occurrence 
of local minima in the MAP estimation. 

\subsection{Workflow}
\begin{enumerate}
\item Select a base model with an acceptable structural component and potentially the most influential stochastic parameters included.
\item If investigating additional ETAs, include in base model but with OMEGA fixed to small value (i.e. 0.00001 FIX) to get derivatives in output dataset.
\item Run linearize run10.mod.
\item Check OFV agreement. Disagreement indicates local minima in MAP estimation (solve by log-transformation to get additive error or set MCETA=1000 in \$ESTIMATION).
\item Add extensions to the linearized model manually and evaluate.
\item When decided which extension to include, implement in standard nonlinear format and reestimate.
\end{enumerate}

If the original model has already been run the initial values will be updated and MAXEVAL=0 will be used for the derivatives model.
If the original model has a phi file this will be used in a \$ETAS record in the linearized model.

\section{Input and options}
\subsection{Required input}
A model file is required on the command line.

\subsection{Optional input}
The following options are valid but intended only for research and method exploration. It is recommended to not use them.
\begin{optionlist}
\optname{epsilon}
Default set. Linearize with respect to epsilon. Disable with -no-epsilon.
\nextopt
\optdefault{error}{add | prop | propadd}
Only relevant if -no-epsilon is set. Use an approximate linearization of the error model instead of an exact.\\
Alternatives are;
\begin{itemize}
\item add (for additive
\item prop (for proportional)
\item propadd (for proportional plus additive)
\end{itemize}
The error model must be defined in a particular way when this option is used. See the scm\_userguide.pdf for details.
\nextopt
\optname{foce}
Default set. Expand around conditional ETA estimates instead of around ETA=0.  
\nextopt
\optname{keep\_covariance}
The default setting will delete \$COVARIANCE from the bootstrap models, to save run time. If option -keep\_covariance is set, PsN will instead keep \$COVARIANCE.
\nextopt
\optname{nointer}
Default off. Don't use interaction. 
\nextopt
\end{optionlist}

\subsection{Some important common PsN options}
There are many options that govern how PsN manages NONMEM runs, and those options are common to all PsN programs that run NONMEM. For a complete list of such options see common\_options.pdf, or psn\_options -h on the command line. A selection of
the most important common options is found here.
\begin{optionlist}
\optname{h or -?}
Print a list of available options and exit. 
\nextopt
\optname{help}
With -help all programs will print a longer help message. If an option name is given as argument, help will be printed for this option. If no option is specified, help text for all options will be printed. 
\nextopt
\optdefault{clean}{integer}
Default is 1. The clean option can take six different values:  
\begin{description}
	\item[0] Nothing is removed 
	\item[1] NONMEM binary and intermediate files except INTER are removed, and files specified with option -extra\_files. 
	\item[2] model and output files generated by PsN restarts are removed, and data files in the NM\_run directory. 
	\item[3] All NM\_run directories are completely removed. If the PsN tool has created modelfit\_dir:s inside the main run directory, these  will also be removed. 
	\item[4] All NM\_run directories and all m1 directories are completely removed.
    \item[5] The entire run directory is removed. This is only useful for execute. The lst-file will be copied even if the run failed.
\end{description}
\nextopt
\optdefault{directory}{string}

Default \guidetoolname\_dirN,
where N will start at 1 and be increased by one each time you run the script. The directory option sets the directory in which PsN 

will run NONMEM and where PsN-generated output files will be stored. You do not have to create the directory,  it will be done for you. If you set -directory to a the name of a directory that already exists, PsN will run in the existing directory, except for scm, boot\_scm and xv\_scm that cannot be started in an existing directory.
\nextopt
\optname{model\_subdir}
	Use an alternative directory structure for PsN. An extra directory
    level unique to each model is introduced between the calling
    directory and the rundirectory. More information about this option can
    be found in PsN.pdf.
\nextopt
   
\optdefault{nm\_version}{string}
Default is 'default'. 
If you have more than one NONMEM version installed you can use option -nm\_version to choose which one to use, as long as it is 

defined in the [nm\_versions] section in psn.conf, see psn\_configuration.pdf for details. You can check which versions are defined, without opening psn.conf, using the command

\begin{verbatim}
psn -nm_versions
\end{verbatim}
\nextopt
\optdefault{seed}{string}
You can set your own random seed to make PsN runs reproducible. The random seed is a string, so both -seed=12345 and -seed=JustinBieber are valid. It is important to know that because of the way the Perl pseudo-random number generator works, for two similar string seeds the random sequences may be identical. This is the case e.g. with the two different seeds 123 and 122. From limited tests it seems as if the final character is ignored and a work around to be sure to set different seeds would be to add a dummy final character. 
Setting the same seed guarantees the same sequence, but setting two slightly different seeds does not guarantee two different random sequences, that must be verified.
\nextopt

\optdefault{threads}{integer}

Default is 5 (if the default psn.conf is used). Use the threads option to enable parallel execution of multiple models.

This option decides how many models PsN will run at the same time, and it is completely independent of whether the individual models are run with serial NONMEM or parallel NONMEM. If you want to run a single model in parallel you must use options -parafile and -nodes. On a desktop computer it is recommended to not set -threads higher the number of CPUs in the system plus one. 
You can specify more threads, but it will probably not increase the performance. If you are running on a computer cluster, you should consult your system administrator to find out how many threads you can specify. 
\nextopt
\optname{version}
Prints the PsN version number of the tool, and then exit. 
\nextopt
\end{optionlist}


\section{Output}
Linearized model file with name run10\_linbase.mod and data file with name run10\_linbase.dta.

\references

\end{document}
