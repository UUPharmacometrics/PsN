\documentclass[12pt]{article}
\usepackage[a4paper,margin=3cm,footskip=2cm]{geometry}
\usepackage{lmodern}
\usepackage[utf8]{inputenc}
\usepackage[T1]{fontenc}
\usepackage{textcomp}
\usepackage{verbatim}
\usepackage{enumitem}
\usepackage{longtable}
\usepackage{alltt}
\usepackage{ifthen}
\usepackage[backend=biber, maxbibnames=99, defernumbers, sorting=none]{biblatex}
\addbibresource{PsN.bib}
% Reduce the size of the underscore
\usepackage{relsize}
\renewcommand{\_}{\textscale{.7}{\textunderscore}}

\input{inputs/version.tex}

\newcommand{\revisiondate}[1]{
\date{#1}
}
\newcommand{\guidetitle}[2]{
\title{#1\\ \vspace{2 mm} {\large PsN \psnversion}}
\date{Revised #2}
}

\newcommand{\references}{
    \printbibliography
}

\newcommand{\doctitle}[2]{
\title{#1}
\date{#2}
}


\newenvironment{optionlist}{
\renewcommand{\arraystretch}{1.1}
\setlength{\leftmargini}{2.5cm}
\begin{description}
%\setlength{\itemsep}{0ex}
}
{\end{description}}

\newcommand{\optname}[1]{\item{{\bfseries\texttt-#1}\newline}}
\newcommand{\optdefault}[2]{\item{{\bfseries\texttt-#1}{\mbox{ = \it #2}}\newline}}
% optconfig is for options that can only be set in a (scm) config file, always without -
\newcommand{\optconfig}[2]{\item{{\bfseries\texttt#1}{\mbox{ = \it #2}}\newline}}

\newcommand{\nextopt}{}


\hyphenation{NONMEM}
\hyphenation{INPUT}

\guidetitle{SIMEVAL user guide}{2016-10-24}

\begin{document}

\maketitle
\newcommand{\guidetoolname}{simeval}


\section{Overview}

The simeval program is a tool for simulation-evaluation based diagnostics that aims to identify model misspecification and outliers.
%based on objective functiuon value (OFV), individual function value (iOFV)
%and empirical Bayes estimates (EBEs). 
The principle is 
\begin{enumerate}
\item Simulate N data sets from final estimates of the model
\item Apply final model to each data set
(simple evaluation given the population final parameter estimates or full estimation with
population parameters that are re-estimated for each simulated data set)
\item Extract output for diagnostics
\begin{description}
\item[pOFV] population objective function value
\item[iOFV] individual objective function value
\item[EBE] parameter individual estimates (empirical bayesian estimates)
\item[IWRES] individual weighted residual
\item[CWRES] conditional weighted residual
\end{description}
\item Provide reference distributions for ”observed” pOFV, iOFV, EBE, CWRES and IWRES (calculated on the real dataset)
\end{enumerate}
The program computes normalized prediction distribution errors for iOFV, EBE, CWRES and IWRES 
using the procedure described in \cite{Comets}, with example use in \cite{Keizer} and \cite{Largajolli}.
It also performs further diagnostics based on posterior predictive check (PPC) principle that are later on introduced in the section Output. Diagnostics plots are created via the -rplots option, see section 
Auto-generated R-plots from PsN.

Example call
\begin{verbatim}
simeval run1.mod -rplots=1
\end{verbatim}
Note: If any IDs in the data set do not have any observation records, then those IDs must be
IGNOREd in \$DATA, otherwise simeval will not produce all output.

\section{Input and options}

\subsection{Required input}

A model file is required on the command-line.

\subsection{Optional input}

\begin{optionlist}
\optdefault{lst\_file}{filename}
Default not set. By default PsN will, before copying the input model to the simulation models, look for an output file with final estimates for the input model and if found update the initial estimates in the simulation models. If option -lsf\_file is set Psn will instead use the final estimates in the given file. If option is not set and no output file is found PsN will not update the estimates but instead rerun the input model, if option -estimate\_input is set. 
\nextopt
\optname{estimate\_input}
Default set. By default, PsN will rerun the input model to get parameter estimates unless an output file is found or option -lst\_file is set. But if option -estimate\_input is unset with -no-estimate\_input and no lst-file is found then the parameter estimates set in the input model will be used for simulations. 
\nextopt
\optdefault{samples}{N}
Default 300. The number of simulations and reestimations/evaluations to run. 
\nextopt
\optdefault{n\_simulation\_models}{N}
Default equal to the number of threads.
By setting this option to a number N greater than 1, the 'samples' simulations and evaluations
will be split equally between N model files, which can be run in parallel.
\nextopt
\optname{reminimize}
Default not set. By default, simulated datasets will be run with MAXEVAL=0 (or equivalent for non-classical estimation methods). If option -reminimize is set then \$EST will be the same as in the input model. We advise the user to use together with MAXEVAL=0 the MCETA option in order to avoid local minima at the individual level.
\nextopt
\optdefault{gls\_data\_file}{filename}
Default gls\_data.dta. A file with input data for the gls program is always generated. This option changes the name of that file. After this file is created, the gls program can be run with option \mbox{-gls\_model} and -ind\_shrinkage with minimum runtime, see gls\_userguide.pdf 
\nextopt
\optdefault{extra\_variables}{comma-separated list}
Default empty. A comma-separated list of extra variables to add to \$TABLE of simulation models.
\nextopt
\optdefault{idv}{variable}
Default TIME. Independent variable for DV vs idv plot.
\nextopt
%\optdefault{iov}{list}
%TODO: list of iov etas, to be treated from all other etas assumed to be iiv. 
%\nextopt
\end{optionlist}

\subsection{Some important common PsN options}
For a complete list see common\_options.pdf, 
or psn\_options -h on the command line.
\begin{optionlist}
\optname{h or -?}
Print the list of available options and exit. 
\nextopt
\optname{help}
With -help all programs will print a longer help message. 
If an option name is given as argument, help will be printed for this option. 
If no option is specified, help text for all options will be printed. 
\nextopt
\optdefault{directory}{'string'}
Default \guidetoolname\_dirN,
where N will start at 1 and
be increased by one each time you run the script. The directory option sets the directory in which PsN 
will run NONMEM and where PsN-generated output files will be stored. 
You do not have to create the directory,  it will be done for you. If you set
-directory to a the name of a directory that already exists, PsN will run in the existing directory.
\nextopt
\optdefault{seed}{'string'}
You can set your own random seed to make PsN runs reproducible.
The random seed is a string, so both -seed=12345 and -seed=JustinBieber are valid.
It is important to know that because of the way the Perl pseudo-random
number generator works, for two similar string seeds the random sequences may be identical. 
This is the case e.g. with the two different seeds 123 and 122. 
Setting the same seed guarantees the same sequence, but setting two slightly different 
seeds does not guarantee two different random sequences, that must be verified.
\nextopt
\optdefault{clean}{'integer'}
Default 1. The clean option can take four different values:  
\begin{description}
\item[0] Nothing is removed 
\item[1] NONMEM binary and intermediate files except INTER are removed, and files specified with option -extra\_files. 
\item[2] model and output files generated by PsN restarts are removed, and data files in the NM\_run directory, and (if option -nmqual is used) the xml-formatted NONMEM output. 
\item[3] All NM\_run directories are completely removed. If the PsN tool has created modelfit\_dir:s inside the main run directory, these  will also be removed. 
\end{description}
\nextopt
\optname{zip}
If this option is set the m1 folder in the run directory will be zipped into the file m1.zip and removed. This will save space and reduce the number of files generated by PsN. PsN can seamlessly handle zipped m1 folders for example when rerunning a command in the same directory using the -directory option.
\nextopt
\optdefault{nm\_version}{'string'}
Default is 'default'. 
If you have more than one NONMEM version installed you can use option
-nm\_version to choose which one to use, as long as it is 
defined in the [nm\_versions] section in psn.conf, see psn\_configuration.pdf for details. 
You can check which versions are defined, without opening psn.conf, using the command
\begin{verbatim}
psn -nm_versions
\end{verbatim}
\nextopt
\optdefault{threads}{'integer'}
Default 5 (if default PsN config file is used). 
Use the threads option to enable parallel execution of multiple models.
This option decides how many models PsN will run at the same time, and it is completely
independent of whether the individual models are run with serial NONMEM or parallel NONMEM.
If you want to run a single model in parallel you must use options -parafile and -nodes.
On a desktop computer it 
is recommended to not set -threads higher the number of CPUs in the system plus one. 
You can specify more threads, 
but it will probably not increase the performance. If you are running on a computer cluster, 
you should consult your 
system administrator to find out how many threads you can specify. 
\nextopt
\optname{version}
Prints the PsN version number of the tool, and then exit. 
\nextopt
\optname{citations}
Print a list of references for this tool. The list will be in BibTeX format.
\nextopt
\end{optionlist}


\subsection{Diagnostic plots}
\newcommand{\rplotsconditions}{
See section Output, subsections Basic and Extended diagnostic plots,
for descriptions of the default simeval plots.
The default simeval template 
requires that libraries gridExtra, PEIP and PerformanceAnalytics are installed.
If the conditions are not fulfilled then no pdf will be generated,
see the .Rout file in the main run directory for error messages.}

PsN can automatically generate R plots to visualize results, if a template file is available.
The PsN installation package includes default template files for a number of tools,
but R plots can be generated for any tool if the user provides the template file.
PsN will create a preamble with some run specific information, 
such as the name of the raw\_results file, the parameter labels and the name of
the tool results file,
and then append 
the template file.
If the template file contains an old preamble, then that preamble will be replaced with 
the new one. This means
the user can modify an old PsN-generated R script and then use this script as a new template,
without having to remove the preamble.
When a template file is available for the tool and option \mbox{-rplots} is 0 or positive, 
the R script will be generated and saved in the main
run directory. 
If R is configured in psn.conf or command 'R' is available and option -rplots is positive 
the script will also be run and a number of pdf-format plots be created.

\rplotsconditions

\begin{optionlist}
\optdefault{rplots}{level}
-rplots<0 means R script is not generated\\ 
-rplots=0 (default) means R script is generated but not run\\ 
-rplots=1 means basic plots are generated\\													  
-rplots=2 means basic and extended plots are generated\\													  
\nextopt
\optdefault{template\_file\_rplots}{file}
When the rplots feature is used, the default template file PsN will use is 
\guidetoolname\_default.R. 
The user can choose a different template file
by setting option -template\_file\_rplots to a different file. 
PsN will first look for the file relative to the current working directory, 
and after that in the -template\_directory\_rplots directory.
\nextopt
\optdefault{template\_directory\_rplots}{path}
PsN can look for the rplots template file in a number of places. The priority order is the
following:
\begin{enumerate}
\item template\_directory\_rplots set on command-line 
\item calling directory (where PsN is started)
\item template\_directory\_rplots set in psn.conf 
\item R-scripts subdirectory of the PsN installation directory
\end{enumerate}
\nextopt
\optdefault{subset\_variable\_rplots}{variable name}
Default not set. The user can specify a subset variable to be used with the -rplots feature. 
This variable
will, if set, be set as subset.variable in the preamble,
and can then be used in the plotting code. 
\nextopt
\end{optionlist}



\section{Output}

\subsection{Basic diagnostic plots}
Basic diagnostic rplots will be generated 
if option -rplots is set >0.

\subsubsection{Population and individual OFV}
\noindent The file PsN\_OFV\_plots.pdf contains
\begin{enumerate}
\item A posterior predictive check of population OFV. The plot shows
observed pOFV (pOFVobs) and the reference
distribution of pOFV based on simulated data (pOFVsim). If the pOFVobs stands outside the pOFVsim distribution the diagnostic is picking up a misspecification. Note that 
this is a global model diagnostic
that can identify major model misspecification but not which part of the model
is inadequate.
\item A histogram of npde of individual OFV (iOFV), together with the expected
normal distribution. If there is no model misspecification, the npde iOFV should
be normally distributed with mean 0 and variance 1.
\item PPC of individual OFV for outlier individuals, i.e.
individuals where observed OFV (iOFVobs) is above entire
distribution of its corresponding simulated individual OFV. If no outliers are found then there will be no individual OFV PPC plots.
\item individual OFV residuals are calculated in the following way:
$RES_{iOFVjk}=\frac{(iOFVobs_{j} - iOFVsim(k)_{j})}{std(iOFVsim(k)_{j})}$
where k=1,...,N and N is the number of simulations and where j=1,...,L and L is the number of subjects. 
Misspecification in the model can be detected if the distribution of $RES_{IOFVj}$ is not uniformly split above and below the zero line. Positive $RES_{IOFVj}$ indicate worse fit of the observed with respect to the simulated reference distribution. Potential outliers can be detected for those IDs with $RES_{IOFVj}$ exceeding the +3SD line. 
\end{enumerate}

\subsubsection{EBE npde}
\noindent The file PsN\_ebe\_npde\_plots.pdf contains
\begin{enumerate}
\item A table with mean and variance of npde of each EBE. If there is no model
misspecification the npde should have mean 0 variance 1.
\item A correlation chart with histograms of individual EBE npde and their
correlations. The expectation around the EBE npde is no correlations, high correlations indicate model misspecification in the omega matrix declaration.
\item A table of the individuals that are outliers with respect to their set of EBE npde and consequently also with respect to their set of EBE. The criteria to define an outlier individual with respect to its EBE is based on the calculation of the following robust multivariate distance for each individual:
$(d_{j})^2=\frac{(EBEnpde_{j} - E[EBEnpde])^2}{var(EBEnpde)}$
where j=1,...,L and L is the number of subjects. Note that the mean and the variance of the EBE npde is known a priori as they follow N(0,1) and this makes the calculation of the distance robust as no estimation is needed. The multivariate distance of each individual follows a chi-squared distribution and this allows us to determine outlier individuals and discern them from simply the extreme ones in the distribution.
\end{enumerate}

\subsubsection{GOF vpc:s}
\noindent The file PsN\_simeval\_vpc\_plots.pdf contains
\begin{enumerate}
\item A vpc of observations vs population predictions, with automatic binning.
\item A vpc of conditional weighted residuals vs the independent variable (idv), default TIME.
\end{enumerate}


\subsubsection{Residuals}
\noindent The file PsN\_residual\_plots.pdf contains
\begin{enumerate}
\item A histogram of npde of IWRES, together with the expected
normal distribution. If there is no model misspecification, the npde should
be normally distributed with mean 0 and variance 1.
\item A histogram of npde of CWRES, together with the expected
normal distribution. If there is no model misspecification, the npde should
be normally distributed with mean 0 and variance 1.
\item A table with mean and variance of IWRES npde and CWRES npde.
\item A table with ID, TIME, DV and PRED for data records where IWRES or CWRES calculated on the observed data lies outside the simulated distribution.
\end{enumerate}


\subsection{Extended diagnostic plots}
Extended diagnostic plots will be generated
if option -rplots is set >1.
In file PsN\_OFV\_plots.pdf will be added
\begin{enumerate}
\item KLD of the individual OFV distribution in the observed dataset (iOFVobs) with respect to the average distribution of the iOFV simulated distribution (iOFVsim). The KLD is a measure of the divergence between two distributions (the iOFVobs and the average iOFVsim). In particular in this exercise we are looking if the same divergence calculated in the observed is reproduced using the simulated dataset. The reference distribution of KLD is constructed by calculating the KLD between each simulated dataset iOFV distribution and the average distribution of the iOFV simulated. If it is not reproduced the same level of divergence a misspecification in the model is present or there are outliers that are driving the difference. 

\end{enumerate}



\subsection{Numeric output}
\begin{description}
\item[ebe\_npde.csv] NPDE of EBE. One row per subject. Columns: ID,
standardized observed EBE (a scalar) where 
$ebe_{standardized}= \\
\left(\overline{ebe}_{obs} - mean (\overline{ebe}_{sim})\right)\left((cov(\overline{ebe}_{sim})\right)^{-1}
\left(\overline{ebe}_{obs} - mean (\overline{ebe}_{sim})\right)^T$,\\ 
NPDE:s for each EBE
\item[summary\_iwres.csv] NPDE of IWRES. One row per data record. Columns: ID, MDV, observed IWRES, NPDE for IWRES, OUTLIER (true or false)
\item[summary\_cwres.csv] NPDE of CWRES. One row per data record. Columns: ID, MDV, observed CWRES, NPDE for CWRES, OUTLIER (true or false)
\item[summary\_iofv.csv] One row per subject. Columns: ID, observed iOFV,
mean of iOFV for evaluation of simulated data,
standard deviation of iOFV for evaluation of simulated data,
standardized observed iOFV where
$iOFV_{standardized}=\frac{(iOFV_{observed} - mean (iOFV_{simulated}))^2}{variance(iOFV_{simulated})}$, NPDE of iOFV
\item[raw\_results\_<model>.csv] e.g. raw\_results\_run1.csv with ofv for evaluation of simulated data
\item[residual\_outliers.csv] A table with ID, TIME, DV, PRED, IWRES and CWRES for records
where observed IWRES or CWRES is outside the simulated distribution.
\end{description}

\subsubsection*{Additional output}
\begin{description}
%\item[decorrelated\_original\_ebe.csv] One row per individual. Columns: ID, decorrelated observed EBE:s
%\item[decorrelated\_original\_iwres.csv] One row per data record. Columns: ID, MDV, decorrelated observed IWRES
\item[raw\_original\_iiv\_ebe.csv]  One row per individual. Columns: ID, observed EBE:s
\item[raw\_original\_iov\_ebe.csv]  One row per individual. Columns: ID, observed EBE:s
\item[raw\_all\_iwres.csv] One row per data record. Columns: ID, MDV, observed IWRES %skip, in summary
\item[raw\_all\_cwres.csv] One row per data record. Columns: ID, MDV, observed IWRES %skip, in summary
\item[raw\_all\_iofv.csv] One row per subject. Columns: ID, observed iOFV, sample.1 iOFV, sample.2 iOFV, etc. 
%\item[ebe\_pde.csv] One row per individual. Columns: ID, PDE of EBE:s
%\item[ebe\_npd.csv] One row per individual. Columns: ID, NPD of EBE:s (normalized PD, not decorrelated)
%\item[iwres\_npd.csv]  One row per data record. Columns: ID, MDV, NPD of IWRES (normalized pd, not decorrelated)
\end{description}


\section{Algorithm overview}

\begin{enumerate}
\item Remove MSFO option from \$EST.
\item Read model file to find set of IIV ETAs. An ETA is assumed to be IIV if the
corresponding \$OMEGA is not SAME
\emph{and} the next \$OMEGA is not SAME either. Also find set of IOV ETAs, grouped by occasions.
An ETA is assumed to belong to IOV occasion 1 if the corresponding \$OMEGA is not SAME,
but the next \$OMEGA is SAME. 
An ETA is assumed to belong to IOV occasion 1+$i$ if the corresponding \$OMEGA the $i$th consecutive SAME \$OMEGA.
\item Check if IWRES is defined in model code. If it is not defined, exclude
IWRES from all \$TABLE listed below, and skip iwres reading and analysis.
\item If \$SIM not present, add a basic \$SIM with a seed and, if \$PRIOR in the model, TRUE=PRIOR.
Create 'n\_simulation\_models' copies of modified
input model with order number of copy indicated in filename. In each copy set unique
seed in \$SIM and set
NSUB so that sum of simulations is equal to 'samples', remove old \$TABLE if present, 
if option reminimize is not set then set MAXEVAL=0, and
add \$TABLE ID DV MDV CWRES IPRED IWRES NOPRINT ONEHEADER NOAPPEND FILE=sim\_res\_table-$\langle$ordernum$\rangle$.dta. 
\item In 'original' input model:
Remove \$SIM if present.
Add \$TABLE $\langle$all undropped items in
\$INPUT$\rangle$ IPRED PRED NOPRINT ONEHEADER NOAPPEND FILE=orig\_pred.dta.
Add \$TABLE ID DV MDV CWRES IPRED PRED IWRES NOPRINT ONEHEADER NOAPPEND FILE=original\_res\_table.dta. 
\item run modified original input model and 'n\_simulation\_models' sim models.
\item Read all iwres\_$\langle$order number$\rangle$.dta files,
storing IWRES values per data point. Compute, per data point, ISHR\_ij=1-sd(IWRES\_ij).
Open orig\_pred.dta, append ISHR column with
computed values,
and print to gls\_input.dta. Print also shrinkage column to new file ind\_iwres\_shrinkage.dta. File gls\_input.dta
can be used as input when running gls program, see gls\_userguide.pdf. 
\item Compute npde for IWRES, CWRES according to \cite{Comets}.
\item Read ETA values from 
phi-files for original and simulated data, and compute npde for EBE:s according to \cite{Comets}.
Perform computations separately for IIV ETA and each occasion IOV ETA.
\item Run basic vpc:s for DV vs PRED and CWRES vs TIME (or idv).
\end{enumerate}

\subsection{Handling of zero-valued EBE:s}
Each EBE column in the phi-file from estimation of the original data set is checked for the existence of
zero-valued EBEs.
The EBE npde computation is performed for each individual in turn. If one or several EBE:s for
an individual is exactly 0 in the original data phi-file then those EBE:s are excluded from the npde computation,
and the corresponding EBE npde for that individual are set to NA.

\subsection{Handling of IWRES}
The simeval program will check the model code for a line that defines IWRES.
If not found then the IWRES processing will be skipped.
If is IWRES defined then IWRES will be read per observation (as checked with MDV==0)
and npde:s computed for each observation.
The file summary\_iwres.csv contains columns with ID,MDV,IWRES,npde\_IWRES,
where IWRES and npde-columns will be empty if MDV-column not 0.

\references

\end{document}
