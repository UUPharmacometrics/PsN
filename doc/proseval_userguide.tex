\documentclass[12pt]{article}
\usepackage[a4paper,margin=3cm,footskip=2cm]{geometry}
\usepackage{lmodern}
\usepackage[utf8]{inputenc}
\usepackage[T1]{fontenc}
\usepackage{textcomp}
\usepackage{verbatim}
\usepackage{enumitem}
\usepackage{longtable}
\usepackage{alltt}
\usepackage{ifthen}
\usepackage[backend=biber, maxbibnames=99, defernumbers, sorting=none]{biblatex}
\addbibresource{PsN.bib}
% Reduce the size of the underscore
\usepackage{relsize}
\renewcommand{\_}{\textscale{.7}{\textunderscore}}

\input{inputs/version.tex}

\newcommand{\revisiondate}[1]{
\date{#1}
}
\newcommand{\guidetitle}[2]{
\title{#1\\ \vspace{2 mm} {\large PsN \psnversion}}
\date{Revised #2}
}

\newcommand{\references}{
    \printbibliography
}

\newcommand{\doctitle}[2]{
\title{#1}
\date{#2}
}


\newenvironment{optionlist}{
\renewcommand{\arraystretch}{1.1}
\setlength{\leftmargini}{2.5cm}
\begin{description}
%\setlength{\itemsep}{0ex}
}
{\end{description}}

\newcommand{\optname}[1]{\item{{\bfseries\texttt-#1}\newline}}
\newcommand{\optdefault}[2]{\item{{\bfseries\texttt-#1}{\mbox{ = \it #2}}\newline}}
% optconfig is for options that can only be set in a (scm) config file, always without -
\newcommand{\optconfig}[2]{\item{{\bfseries\texttt#1}{\mbox{ = \it #2}}\newline}}

\newcommand{\nextopt}{}


\hyphenation{NONMEM}
\hyphenation{INPUT}

\guidetitle{PROSEVAL user guide}{2016-08-31}

\newcommand{\guidetoolname}{proseval}

\begin{document}

\maketitle


\section{Introduction}
Prospective evaluation is a tool that runs successive evaluations on a model each with an increased number of observations per individual.

Examples
\begin{verbatim}
proseval run28.mod
\end{verbatim}

\section{Input and options}

\subsection{Required input}
A model file

%\subsection{Optional input}

%\begin{optionlist}
%\optdefault{samples}{n}
%Number of simulated datasets to generate. Default is 100.
%\nextopt
%\optname{models}
%If this option is present on the command line a list of model files can be passed as arguments instead of an scm logfile. For example \verb|pvar -models run1.mod run2.mod -parameters=CL,V|
%\end{optionlist}

\section{Result}

The result.csv contains these columns: ID, TIME, EVID, WRES, IWRES and OBS where
OBS is the number of observations that was used for evaluating that line in the results.


%\references

\end{document}
