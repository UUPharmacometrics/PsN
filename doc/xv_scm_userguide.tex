\documentclass[12pt]{article}
\usepackage[a4paper,margin=3cm,footskip=2cm]{geometry}
\usepackage{lmodern}
\usepackage[utf8]{inputenc}
\usepackage[T1]{fontenc}
\usepackage{textcomp}
\usepackage{verbatim}
\usepackage{enumitem}
\usepackage{longtable}
\usepackage{alltt}
\usepackage{ifthen}
\usepackage[backend=biber, maxbibnames=99, defernumbers, sorting=none]{biblatex}
\addbibresource{PsN.bib}
% Reduce the size of the underscore
\usepackage{relsize}
\renewcommand{\_}{\textscale{.7}{\textunderscore}}

\input{inputs/version.tex}

\newcommand{\revisiondate}[1]{
\date{#1}
}
\newcommand{\guidetitle}[2]{
\title{#1\\ \vspace{2 mm} {\large PsN \psnversion}}
\date{Revised #2}
}

\newcommand{\references}{
    \printbibliography
}

\newcommand{\doctitle}[2]{
\title{#1}
\date{#2}
}


\newenvironment{optionlist}{
\renewcommand{\arraystretch}{1.1}
\setlength{\leftmargini}{2.5cm}
\begin{description}
%\setlength{\itemsep}{0ex}
}
{\end{description}}

\newcommand{\optname}[1]{\item{{\bfseries\texttt-#1}\newline}}
\newcommand{\optdefault}[2]{\item{{\bfseries\texttt-#1}{\mbox{ = \it #2}}\newline}}
% optconfig is for options that can only be set in a (scm) config file, always without -
\newcommand{\optconfig}[2]{\item{{\bfseries\texttt#1}{\mbox{ = \it #2}}\newline}}

\newcommand{\nextopt}{}


\hyphenation{NONMEM}
\hyphenation{INPUT}

\guidetitle{XV\_SCM user guide}{2017-09-22}

\begin{document}

\maketitle


\section{Overview}
The xv\_scm program is an implementation of the method described in \cite{Katsube1} and \cite{Katsube2} 

The Cross-validated Stepwise Covariate Model (xv\_scm) building tool of PsN implements cross-validated model 
size selection for acovariate model. It relies on the SCM tool of PsN.        
A configuration file must be written for each xv\_scm run. The format of the configuration file follows the format 
of the scm configuration file exactly, except that options  search\_direction, gof, p\_value, p\_forward, p\_backward 
and  update\_derivatives are ignored.

Example xv\_scm call:
\begin{verbatim}
xv_scm -config_file=config_xv.scm -groups=5 -splits=3 -seed=12345
\end{verbatim}

\section{Input and options}

\subsection{Required input}
Name of an scm configuration file, including search path if file is not stored in current directory. 
The configuration file may contain all scm command-line options and some optional PsN common options.
If a common option that is *not* listed here is set in the scm configuration file then that option 
will be ignored.
The following common options, described in common\_options.pdf,can be set in the configuration file: 

	\mbox{nm\_version}, \mbox{nmfe\_options}, \mbox{abort\_on\_fail}, \mbox{compress}, \mbox{directory}, 
	\mbox{extra\_data\_files}, \mbox{extra\_files}, \mbox{picky}, \mbox{retries}, \mbox{threads} and 
	\mbox{tweak\_inits}. 
	
As of PsN-3.4.3 it is possible to skip the option name for the configuration file, and call scm with 
scm config.scm instead of the traditional scm -config_file=config.scm
\subsection{Optional input}

These options are specific to xv\_scm, and they can only be given on the command-line, not in the configuration file.

\begin{optionlist}
\optdefault{config\_file} 
Name of an scm configuration file, including search path if file is not stored in current directory. 
The configuration file may contain all scm command-line options and some optional PsN common options.
If a common option that is *not* listed here is set in the scm configuration file then that option 
will be ignored.
The following common options, described in common_options.pdf,can be set in the configuration file: 
	\mbox{nm_version}, \mbox-nmfe_options}, \mbox{abort_on\_fail}, \mbox{compress}, \mbox{directory}, 
	\mbox{extra\_data\_files}, \mbox{extra\_files}, \mbox{picky}, \mbox{retries}, \mbox{threads} and 
	\mbox{tweak\_inits}. 
	
As of PsN-3.4.3 it is possible to skip the option name for the configuration file, and call scm with
scm config.scm 
instead of the traditional
scm -config_file=config.scm
\nextopt
\optdefault{derivatives\_data} 
Only relevant with linearize method. Give derivatives data as table input
rather than letting the scm run the nonlinear model to obtain them. Saves time.
\nextopt
\optdefault{do\_not\_drop} 
Default empty string. Since the number of columns are restricted to 20 for NONMEM (50 for NM7) 
it is necessary to minimize the number of undropped columns. The scm utility uses the '=DROP' 
syntax of NONMEM to exclude the covariate columns that are not currently tested in a specific 
candidate model. If some covariates are used in e.g. PK, PRED or TABLE in the basic model you must 
list them using the do\_not\_drop option to prevent them from being dropped. If the -linearize 
option is used, do\_not\_drop has a different usage, see the section on the linearization method. 
\nextopt
\optdefault{epsilon} 
Only relevant with linearize method. Linearize with respect to epsilons,
set by default. Disable with -no-epsilon.
\nextopt
\optdefault{error} 
Only relevant with linearize method, and only if -no-epsilon is set. 
Use an approximated linearization of the error model instead of an exact.
Alternatives are add (for additive), prop (for proportional),
propadd (for proportional plus additive) or user (for user defined).
The error model must be defined in a particular way when this option is used,
see the scm userguide for details.
\nextopt
\optdefault{foce} 
Default set. Only relevant with linearize method. Expand around 
conditional ETA estimates instead of around ETA=0.    
\nextopt
\optdefault{global_init}
Default is 0.001. With global_init option the initial estimates of parameters
in covariate parameterizations are set to global\_init. If using inits section 
in configuration file individual initial values are used instead of one global.
\nextopt
\optdefault{groups}{N}
Default 5. The number of cross-validation groups to make an N-fold cross-validation. 	
\nextopt
\optdefault{linearize} 
Add covariate relations to a linearized version of the input model instead of
to the original nonlinear model.
\nextopt
\optdefault{logfile}
Default scmlog.txt. The name of the logfile.
\nextopt
\optdefault{lst_file} 
Default not set. Update original model with final estimates from this file 
before running model to obtain derivatives.
\nextopt
\optdefault{max\_steps}{N} 
Do not take more that max\_steps forward steps, even if there are more covariates 
left to add and a significant inclusion was made in the last step.
\nextopt
\optdefault{model}
Default not set. The name of the basic model file, without any parameter-covariate
relations included.
\nextopt
\optdefault{noabort} 
Default not set. Only relevant with the linearize method. If set,
the program will add NOABORT to \$EST of the linearized models.
\nextopt
\optdefault{only\_successful} 
Default not set. Only consider runs with MINIMIZATION SUCCESSFUL 
(or equivalent for non-classical estimation methods) when 
selecting the covariate to add/remove in each step.
\nextopt
\optdefault{parallel\_states} 
Default not set. Instead of trying valid\_states sequentially, try all states after the 
current one simultaneously. It is currently not possible to stop testing a parameter-covariate 
relation that has been included in an lower state if there are higher states defined. 
To achieve this the user has to stop the search and restart with a new configuration file where 
no higher states are tested for the parameter-covariate pair. 
\nextopt
\optdefault{splits}{N} 
Default 1. The number of times to perform a complete cross-validation
with a new data split. 
\nextopt
\optdefault{stratify\_on}{variable} 
PsN will try to preserve the relative proportions of individuals
with different values of the stratification variable
when creating data subsets during cross-validation.
The stratification variable must be found in the original dataset,
and there must be at least 'groups' individuals having each
unique value of the stratification variable. If the stratification
variable is continuous it is recommended to first group the values
and then stratify on group number instead of the continuous variable.
\nextopt
\end{optionlist}

Do not set scm option -only\_successful in xv\_scm. That option would interfere with the xv\_scm algorithm.


\section{Algorithm overview}

For each split: 
\begin{itemize}
\item[] Divide the dataset into 'groups' equally sized subsets, using stratification if option -stratify\_on is set.
\item[] For each data subset: 
\begin{itemize}
\item[] Call the selected subset the prediction/test data and the remaining 'groups'-1 subsets the estimation/training data. 
\item[] Run a regular scm on the estimation data, using the scm input option given on the command-line and the 
configuration file except forcing options search\_direction=forward, p\_forward=1,  gof=p\_value, -no-update\_derivatives. 
For the base model and for the model selected by the scm in each iteration a  prediction run is performed. The prediction 
run is  done by copying the model, updating the initial estimates with the final estimates for the same model based on the 
estimation data, setting MAXEVAL=0 or equivalent for non-classical estimation methods, and running the model with the 
prediction data. The OFV of the prediction run is then collected and reported in output.
If the linearization method is used (option -linearize to scm), then a prediction step is needed also for the derivatives model.
 After running the nonlinear derivatives model on the estimation data, a prediction step is run as above for the derivatives model. 
 Then the derivatives output from the derivatives prediction step replaces the original prediction data in the prediction steps for 
 all the linearized models, including the linearized base model.
\end{itemize}
\end{itemize}
\section{Output}
The files xv\_ofv\_results.csv,  xv\_relation\_rank\_order.csv and xv\_percent\_inclusion\_by\_level.csv contain results and 
summaries of the runs.

\references

\end{document}
