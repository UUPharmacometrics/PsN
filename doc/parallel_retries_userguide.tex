\documentclass[12pt]{article}
\usepackage[a4paper,margin=3cm,footskip=2cm]{geometry}
\usepackage{lmodern}
\usepackage[utf8]{inputenc}
\usepackage[T1]{fontenc}
\usepackage{textcomp}
\usepackage{verbatim}
\usepackage{enumitem}
\usepackage{longtable}
\usepackage{alltt}
\usepackage{ifthen}
\usepackage[backend=biber, maxbibnames=99, defernumbers, sorting=none]{biblatex}
\addbibresource{PsN.bib}
% Reduce the size of the underscore
\usepackage{relsize}
\renewcommand{\_}{\textscale{.7}{\textunderscore}}

\input{inputs/version.tex}

\newcommand{\revisiondate}[1]{
\date{#1}
}
\newcommand{\guidetitle}[2]{
\title{#1\\ \vspace{2 mm} {\large PsN \psnversion}}
\date{Revised #2}
}

\newcommand{\references}{
    \printbibliography
}

\newcommand{\doctitle}[2]{
\title{#1}
\date{#2}
}


\newenvironment{optionlist}{
\renewcommand{\arraystretch}{1.1}
\setlength{\leftmargini}{2.5cm}
\begin{description}
%\setlength{\itemsep}{0ex}
}
{\end{description}}

\newcommand{\optname}[1]{\item{{\bfseries\texttt-#1}\newline}}
\newcommand{\optdefault}[2]{\item{{\bfseries\texttt-#1}{\mbox{ = \it #2}}\newline}}
% optconfig is for options that can only be set in a (scm) config file, always without -
\newcommand{\optconfig}[2]{\item{{\bfseries\texttt#1}{\mbox{ = \it #2}}\newline}}

\newcommand{\nextopt}{}


\hyphenation{NONMEM}
\hyphenation{INPUT}

\guidetitle{RANDTEST user guide}{2015-02-12}


\begin{document}

\maketitle


\section{Introduction}
The randtest script is a PsN tool that allows you to run 
multiple copies of
a single model with tweaked initial estimates in parallel.
In PsN terminology it is called a retry to (re)run a model with tweaked 
initial estimates.

The retries (see documentation in common\_options\_defaults\_versions\_psn.pdf) 
are always done serially in execute and other PsN programs. 
The parallel\_retries script is a help script that takes a single model 
as input, creates 'min\_retries' extra copies of this model 
with tweaked initial estimates, and then runs the original plus 
the extra models in parallel in separate NM\_run subdirectories. 
Results are summarized in raw\_results.csv in the run directory. 
The script does not select the best retry, but the user must do that manually 
based on the raw\_results csv-file. The script parallel\_retries accept all
options in common\_options\_defaults\_versions\_psn.pdf. 

Example
\begin{verbatim}
parallel_retries run33.mod -min_retries=4 -threads=5 -seed=12345
\end{verbatim}

\section{Input and options}
\subsection{Required input}
A model file is required on the command-line. 
In addition, either -min\_retries or -rawres\_input is required, see below.

\subsection{Optional input}
All options listed in document common\_options\_defaults\_versions\_psn.pdf apply to parallel\_retries. 
Those options govern how the NONMEM runs are managed, and hence apply to all PsN scripts.
Since parallel\_retries does not do much except running a model with NONMEM, 
it has a very short list of additional options:

\begin{optionlist}
\optdefault{min\_retries}{integer}
The number of model copies with tweaked initial estimates to run. 
If this option is not used then -rawres\_input must be set.
\nextopt
\optdefault{rawres\_input}{filename}
Default not used. If this option is not used then -min\_retries must be used.
An alternative way to estimate with different initial estimates. 
Instead of using initial estimates from a random perturbation, take parameter initial values from a raw\_results-like file.
The raw results file must contain at least as many samples as the input -samples to parallel\_retries, the labels for  THETA/OMEGA/SIGMA 
in the file must match the labels in the simulation model given as input to parallel\_retries, 
the theta columns must be directly followed by the omega columns which must be directly followed by the sigma columns, and the first column 
must have header model just as a bootstrap raw\_results file. Note that is is possible to generate a file with initial parameter estimates outside of PsN, 
as long as the file follows the format rules.
\nextopt
\optdefault{samples}{N}
Only valid in combination with -rawres\_input, see above. The number of
parameter sets to use from the rawres\_input file. 
Default is to use all sets after skipping the first 'offset\_rawres' sets.
\nextopt
\optdefault{offset\_rawres}{N}
Only valid in combination with -rawres\_input, see above. Default 1.
The number of result lines to skip in the input raw results file before starting to read final parameter estimates. In a regular
bootstrap raw\_results file the first line of estimates refers to the input model with the full dataset, so therefore the default offset is 1.
\nextopt
\optname{model\_dir\_name}
Default not used. This option changes the basename of the run directory from 
parallel\_retries\_dir to $\langle$modelfile$\rangle$.dir. where $\langle$modelfile$\rangle$ 
is the name of the (first) input model file, without the extension. 
The directories will be numbered starting from 1, increasing the number each time parallel\_retries 
is run with a model file with the 
same name. If the option directory is used it will override -model\_dir\_name.
\nextopt
\optname{timestamp}
Default not used. This option changes the name of the run directory to $\langle$modelfile$\rangle$-PsN-$\langle$date$\rangle$
where $\langle$modelfile$\rangle$ is the name of the first model file in the list given as arguments, without the extension,
and $\langle$date$\rangle$ is the time and date the run was started. 
Example: directory name run1-PsN-2014-06-12-152502 for a run that was started at 15:25:02 June 12th in year 2014.
If the option directory is used it will override -timestamp.
\nextopt
\optname{prepend\_options\_to\_lst}
Default not used. This option makes PsN prepend the final lst-file (which is copied back to the directory from which parallel\_retries was called) with the file version\_and\_option\_info.txt which contains run information, including     all actual values of optional PsN options. PsN can still parse the lst-file with the options prepended, so the file can still be used it as input to e.g. sumo, vpc or update\_inits. Disabled with -no-prepend\_options\_to\_lst if set in psn.conf.
\nextopt
\end{optionlist}

\section{Output}
When the NONMEM runs are finished, the raw\_results file in the
main run directory will contain a summary of the results of the 
retries.



\end{document}
