\documentclass[12pt]{article}
\usepackage[a4paper,margin=3cm,footskip=2cm]{geometry}
\usepackage{lmodern}
\usepackage[utf8]{inputenc}
\usepackage[T1]{fontenc}
\usepackage{textcomp}
\usepackage{verbatim}
\usepackage{enumitem}
\usepackage{longtable}
\usepackage{alltt}
\usepackage{ifthen}
\usepackage[backend=biber, maxbibnames=99, defernumbers, sorting=none]{biblatex}
\addbibresource{PsN.bib}
% Reduce the size of the underscore
\usepackage{relsize}
\renewcommand{\_}{\textscale{.7}{\textunderscore}}

\input{inputs/version.tex}

\newcommand{\revisiondate}[1]{
\date{#1}
}
\newcommand{\guidetitle}[2]{
\title{#1\\ \vspace{2 mm} {\large PsN \psnversion}}
\date{Revised #2}
}

\newcommand{\references}{
    \printbibliography
}

\newcommand{\doctitle}[2]{
\title{#1}
\date{#2}
}


\newenvironment{optionlist}{
\renewcommand{\arraystretch}{1.1}
\setlength{\leftmargini}{2.5cm}
\begin{description}
%\setlength{\itemsep}{0ex}
}
{\end{description}}

\newcommand{\optname}[1]{\item{{\bfseries\texttt-#1}\newline}}
\newcommand{\optdefault}[2]{\item{{\bfseries\texttt-#1}{\mbox{ = \it #2}}\newline}}
% optconfig is for options that can only be set in a (scm) config file, always without -
\newcommand{\optconfig}[2]{\item{{\bfseries\texttt#1}{\mbox{ = \it #2}}\newline}}

\newcommand{\nextopt}{}


\hyphenation{NONMEM}
\hyphenation{INPUT}

\guidetitle{Extended grid user guide}{2017-10-13}


\begin{document}

\maketitle
\newcommand{\guidetoolname}{extended\_grid}


\section{Introduction}
The extended\_grid program implements the method presented in \cite{Savic}

Example call:
\begin{verbatim}
extended_grid run1.mod
\end{verbatim}

This will create a new directory extended\_grid\_dirX where X is a number increased every time you run the program. Inside that
directory it will create a new model file for each OMEGA in the file. You will be able to retrieve the results from the
corresponding table file.

\section{Input and options}

\subsection{Required input}
A model file is required on the command-line. No other options are required.
	
\subsection{Optional input}	
\begin{optionlist}
\optdefault{dv\_column}{column\_name}
The dependent variable column. Default DV. 
\nextopt
\optdefault{id\_column}{column\_name}
The header of the column marking individuals. Default ID.
\nextopt
\optdefault{ids\_to\_simulate}{N}
The number of individuals to simulate. Default is the number of individuals in the original dataset.   
\nextopt
\optdefault{omega\_multiplier}{number}
Default 1. The initial omegas in the simulation model and extended model will be multiplied by this number. This gives inflation (or deflation if the multiplier \verb|<| 1) of the variance.

\nextopt
\end{optionlist}

\subsection{Some important common PsN options}
For a complete list see common\_options.pdf, 
or psn\_options -h on the commandline.
\begin{optionlist}
\optname{h or -?}
Print a list of available options and exit. 
\nextopt
\optname{help}
With -help all programs will print a longer help message. If an option name is given as argument, help will be printed for this option. If no option is specified, help text for all options will be printed. 
\nextopt
\optdefault{clean}{integer}
Default is 1. The clean option can take six different values:  
\begin{description}
	\item[0] Nothing is removed 
	\item[1] NONMEM binary and intermediate files except INTER are removed, and files specified with option -extra\_files. 
	\item[2] model and output files generated by PsN restarts are removed, and data files in the NM\_run directory. 
	\item[3] All NM\_run directories are completely removed. If the PsN tool has created modelfit\_dir:s inside the main run directory, these  will also be removed. 
	\item[4] All NM\_run directories and all m1 directories are completely removed.
    \item[5] The entire run directory is removed. This is only useful for execute. The lst-file will be copied even if the run failed.
\end{description}
\nextopt
\optdefault{directory}{string}

Default \guidetoolname\_dirN,
where N will start at 1 and be increased by one each time you run the script. The directory option sets the directory in which PsN 

will run NONMEM and where PsN-generated output files will be stored. You do not have to create the directory,  it will be done for you. If you set -directory to a the name of a directory that already exists, PsN will run in the existing directory, except for scm, boot\_scm and xv\_scm that cannot be started in an existing directory.
\nextopt
\optname{model\_subdir}
	Use an alternative directory structure for PsN. An extra directory
    level unique to each model is introduced between the calling
    directory and the rundirectory. More information about this option can
    be found in PsN.pdf.
\nextopt
   
\optdefault{nm\_version}{string}
Default is 'default'. 
If you have more than one NONMEM version installed you can use option -nm\_version to choose which one to use, as long as it is 

defined in the [nm\_versions] section in psn.conf, see psn\_configuration.pdf for details. You can check which versions are defined, without opening psn.conf, using the command

\begin{verbatim}
psn -nm_versions
\end{verbatim}
\nextopt
\optdefault{seed}{string}
You can set your own random seed to make PsN runs reproducible. The random seed is a string, so both -seed=12345 and -seed=JustinBieber are valid. It is important to know that because of the way the Perl pseudo-random number generator works, for two similar string seeds the random sequences may be identical. This is the case e.g. with the two different seeds 123 and 122. From limited tests it seems as if the final character is ignored and a work around to be sure to set different seeds would be to add a dummy final character. 
Setting the same seed guarantees the same sequence, but setting two slightly different seeds does not guarantee two different random sequences, that must be verified.
\nextopt

\optdefault{threads}{integer}

Default is 5 (if the default psn.conf is used). Use the threads option to enable parallel execution of multiple models.

This option decides how many models PsN will run at the same time, and it is completely independent of whether the individual models are run with serial NONMEM or parallel NONMEM. If you want to run a single model in parallel you must use options -parafile and -nodes. On a desktop computer it is recommended to not set -threads higher the number of CPUs in the system plus one. 
You can specify more threads, but it will probably not increase the performance. If you are running on a computer cluster, you should consult your system administrator to find out how many threads you can specify. 
\nextopt
\optname{version}
Prints the PsN version number of the tool, and then exit. 
\nextopt
\end{optionlist}


\section{Output}
	
	File final\_nptab.csv
	
	\section{Known bugs and problems}
	
	No safeguard against infinite squared weighted individual residual in auto-generated code.
	
	\section{Technical overview of algorithm}
	
	\begin{enumerate}
		\item A run directory extended\_grid\_dirN is created (unless option -dir is set to something different).
		\item The input model, without any changes, is run in the orig\_modelfit\_dir subdirectory. It is not checked that the input model does not contain NONP. Improvement: Add check that input model does not contain NONP.
		\item The input model is copied to the simulation model called simulation.mod, along with the original dataset where the copy is called simulation.dta. 
																											   
										 
		\item Modifications in the simulation model:
		\begin{enumerate}
			\item The initial estimates are updated with the final estimates from the input model run.
			\item If option -omega\_multiplier is set: For each OMEGA initial value, replace the initial value 'init' with 'init'*omega\_multiplier. Diagonal and non-diagonal values are treated alike. 
																				
			\item Remove \$ESTIMATION, \$COVARIANCE, \$TABLE.
			\item Set \$SIMULATION ($\langle$seed$\rangle$) ONLYSIM
			Improvement: Add (seed2 UNIFORM) if new option -uniform is set.
			\item Count the number of OMEGAs and set 
			\$TABLE ID ETA1 ETA2 ... ETA$\langle$nOMEGA$\rangle$ FIRSTONLY FILE=sim
			Note: 'ID' in set to option id\_column.
			\item If option -ids\_to\_simulate=N is set: If the number is smaller than or equal to the number of individuals in the dataset then all but the N first individuals are removed from simulation.dta. Else (N is larger than the number of individuals in the dataset) then the first individual in simulation.dta is copied N-$\langle$original number of ids$\rangle$ times and those individuals are appended so that simulation.dta contains N individuals in total. After modifications the IDs are renumbered ascending.
																												  
																												  
																												
																 
			Note: Handling of individuals is not affected by option id\_column, so it is likely that option -ids\_to\_simulate will fail if id\_column is not 'ID'. 
			Improvement: Add support for -id\_column and -ids\_to\_simulate in combination
		\end{enumerate}
		\item The simulation model is run in the simulation\_dir subdirectory.
		\item The input model is copied to the extended grid model called extended.mod, along with the original dataset where the copy is called extended.dta. 
					 
		\item Modifications in the extended grid model:
		\begin{enumerate} 
			\item The initial estimates are updated with the final estimates from the input model run.
			\item If option -omega\_multiplier is set: For each OMEGA initial value, replace the initial value 'init' with 'init'*omega\_multiplier. Diagonal and non-diagonal values are treated alike. 
			\item In extended.dta append a column of zeros and set header 'TYPE'. Also append rows for 'ids\_to\_simulate' (see help for this option) individuals, where each of these appended individuals have nOMEGA rows. On the ith row of extra individual j the DV column (either default DV or set with option dv\_column) is set to the ETAi value for individual j read from simulated file sim (steps 4e and 5 above), the ID column (either default ID or set with option id\_column) is set to j and the TYPE column to i.     
																																			
																															 
																																				  
																											
			\item Append TYPE to \$INPUT
			\item Edit \$ERROR (no warning if \$ERROR not present): Wherever EPS(n) is found replace it with ETA(n+n\_etas\_in\_original\_model). Enclose whole \$ERROR block with 'IF (TYPE.LE.0) THEN' and 'ENDIF' unless a line starting with 'IF' was found in the \$ERROR block. For each ETAi in the original model, append to \$ERROR a line
																																	  
															   
			IF( TYPE.EQ.i ) Y = ETA(i)+ERR(2)*W/100000
			\item For each ETA added in step 7e, add a \$OMEGA with initial value 1
			\item Add \$SIGMA 1 FIX
			\item set option MAXEVAL=0 POSTHOC in \$ESTIMATION (no check that estimation method is classic)
			\item remove \$COVARIANCE
			\item Add \$NONPARAMETRIC MARGINALS MSFO=MSF01 UNCONDITIONAL
			\item in \$PRED/\$PK code block (error message if neither is found) append line
			JD = DEN\_
			and for each ETAi in the original model a line  
			DNi=CDEN\_(i)
			\item Add \$TABLE ID ETA1 ... ETA$\langle$n\_original\_omega$\rangle$ JD DN1 ... DN$\langle$n\_original\_omega$\rangle$ NOPRINT NOAPPEND FIRSTONLY FILE=nptab
							   
		\end{enumerate}
		\item The extended grid model is run in the extended\_dir subdirectory.
		\item Analyze the output in file nptab. 
		\begin{enumerate}
			\item First transform the JD column in two steps. In the first step leave the first $\langle$number of original individuals$\rangle$ rows unchanged, then transform the remaining rows according to new\_value=old\_value-(1/Ntotal) where Ntotal is the total number of rows in the JD column. In the second step transform all rows according to new\_value =old\_value*Ntotal/Noriginal where Noriginl is the number of original individuals.
			\item For each eta column,  sort the ETAs and the transformed JD values as a group in ascending order based on the ETA values. Then create a new column with the cumulative sum of the sorted JD values.
																																				  
																																			   
															 
			\item Print a final output table ‘final\_nptab.csv’ with the transformed JD column followed by all pairs of sorted ETA and cumulative  JD sum columns.
		\end{enumerate}
	\end{enumerate}
	
	\references
	
\end{document}
