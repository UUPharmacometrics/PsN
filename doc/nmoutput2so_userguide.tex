\documentclass[12pt]{article}
\usepackage[a4paper,margin=3cm,footskip=2cm]{geometry}
\usepackage{lmodern}
\usepackage[utf8]{inputenc}
\usepackage[T1]{fontenc}
\usepackage{textcomp}
\usepackage{verbatim}
\usepackage{enumitem}
\usepackage{longtable}
\usepackage{alltt}
\usepackage{ifthen}
\usepackage[backend=biber, maxbibnames=99, defernumbers, sorting=none]{biblatex}
\addbibresource{PsN.bib}
% Reduce the size of the underscore
\usepackage{relsize}
\renewcommand{\_}{\textscale{.7}{\textunderscore}}

\input{inputs/version.tex}

\newcommand{\revisiondate}[1]{
\date{#1}
}
\newcommand{\guidetitle}[2]{
\title{#1\\ \vspace{2 mm} {\large PsN \psnversion}}
\date{Revised #2}
}

\newcommand{\references}{
    \printbibliography
}

\newcommand{\doctitle}[2]{
\title{#1}
\date{#2}
}


\newenvironment{optionlist}{
\renewcommand{\arraystretch}{1.1}
\setlength{\leftmargini}{2.5cm}
\begin{description}
%\setlength{\itemsep}{0ex}
}
{\end{description}}

\newcommand{\optname}[1]{\item{{\bfseries\texttt-#1}\newline}}
\newcommand{\optdefault}[2]{\item{{\bfseries\texttt-#1}{\mbox{ = \it #2}}\newline}}
% optconfig is for options that can only be set in a (scm) config file, always without -
\newcommand{\optconfig}[2]{\item{{\bfseries\texttt#1}{\mbox{ = \it #2}}\newline}}

\newcommand{\nextopt}{}


\hyphenation{NONMEM}
\hyphenation{INPUT}

\guidetitle{NMOUTPUT2SO user guide}{2018-01-31}
\begin{document}

\maketitle

\section{Introduction}
The nmoutput2so tool creates and populates a DDMoRe standard output \cite{Terranova} xml file given the output from one or more NONMEM runs. It is necessary that the format of the model file and other files follow a set of requirements for this to work. \\
The Perl module XML::LibXML is required by nmoutput2so, but not PsN in general. Make sure that it is installed before using nmoutput2so (If you are using ActivePerl on Windows please read the installation section of the PsN homepage:
\texttt{https://uupharmacometrics.github.io/PsN/install.html}.\\ 
Examples:
\begin{verbatim}
nmoutput2so run1.lst run2.lst
\end{verbatim}
This will add the NONMEM output from run1.lst in an SOBlock called run1 and the output from run2.lst to an SOBlock called run2. All information that is possible to extract and that is supported will be added to the generated SO file, which will be called run1.SO.xml. 
\begin{verbatim}
nmoutput2so -bootstrap_results=bootstrap_results.csv
-so_filename=res.SO.xml
\end{verbatim}
This will add the PsN bootstrap results to a lone SOBlock in the file called res.SO.xml.
\section{Input and options}

\subsection{Required input}
One or more NONMEM results files (.lst) and or -bootstrap\_results and or -vpc\_results (see those options) is mandatory. One SOBlock will be created for each .lst file and the bootstrap/vpc results will be added to the first SOBlock or in it 
own block if no .lst files were specified. Wildcards will work (i.e. *.lst).

\subsection{Optional input}

\begin{optionlist}
\optname{bootstrap\_results}
A bootstrap\_results file as output from PsN to add to the SO.
\nextopt
\optname{compress}
Set to automatically gzip compress the output SO file.
\nextopt
\optdefault{error}{string}
Specify a string to be added as an error message in the TaskInformation of the first SOBlock.
\nextopt
\optname{exclude\_elements}
A comma separated list of simple XPaths relative the SOBlock to exclude from the SO.\\
Example:
\begin{verbatim}
-exclude_elements=Estimation/PopulationEstimates
\end{verbatim}
\nextopt
\optname{external\_tables}
Set option to create external table files for the simulation results.
\nextopt
\optname{extra\_output}
A comma separated list of tabulated model variables to add a separate SimulatedProfiles element.
\nextopt
\optname{generate\_pharmml}
Generate and connect a minimal companion PharmML to the SO. This ca be used to automatically extract for example the parameter variability levels.
\nextopt
\optname{include\_fixed\_params}
Set option to include estimates of all fixed parameters.
\nextopt
\optdefault{max\_replicates}{integer}
Default is to add all replicates. Set a maximum number of simulation replicates to add. 
\nextopt
\optdefault{message}{string}
Specify a string to be added as an information message in the TaskInformation of the first SOBlock.
\nextopt
\optname{only\_include\_elements}
A comma separated list of XPaths relative the SOBlock. These elements will be the only ones used.\\
Example: 
\begin{verbatim}
-only_include_elements=Estimation/Predictions
\end{verbatim}
\nextopt
\optname{pharmml}
Set a name of the pharmml file used to generate the SO in the PharmMLRef element.
\nextopt
\optname{pretty}
Default is to not add intentations and thus to generate as compact xml files as possible. Set if SO should be output with indentations and newlines.
\nextopt
\optname{psn\_connector}
Do not use! Only for internal use by the PsN connector.
\nextopt
\optname{rundir}
Use to point to a PsN run directory to automatically extract results from.\\ The following PsN tools are not (yet) supported;\\ bootstrap, sse, vpc, execute and nca. 
\nextopt
\optdefault{so\_filename}{string}
Set a filename for the resulting SO xml file. If this is not set the file stem of the first .lst file will be used.
\nextopt
\optname{so\_version}
Default is 0.3. Set the version of the SO file to create. 
\nextopt
\optdefault{toolname}{string}
Default is NONMEM. The toolname to use for messages. 
\nextopt
\optname{use\_tables}
Default set. Will use table files (sdtab, patab and cotab) to populate the elements of the SO that needs these. \\Use -no-use\_tables to not use any table files.
\nextopt
\optname{verbose}
Set to print information, such as all errors and warnings, during conversion to stdout.
\nextopt
\optdefault{vpc\_results}{filename}
A vpc\_results file as output from PsN to add to the SO.
\nextopt
\end{optionlist}

\section{Requirements on the model file and other files}

\subsection{Requirements on the model file}
\begin{itemize}
    \item It is recommended that all NONMEM parameters (THETAs, OMEGAs and SIGMAs) have labels. A label is a one word (and only one word) comment on the same line as the definition of the initial estimate of the parameter. This means that all block definitions must be written on different rows. Example:
        \begin{verbatim}
$OMEGA BLOCK(2)
 1.2     ; PPV_CL
 0.01    ; CORR_CL_V
 0.1     ; PPV_V
        \end{verbatim}
        Paramaters without labels will get a mangled name i.e. THETA(1) becomes THETA\_1\_
    \item To be able to populate the Residuals and the Predictions entries (or SimulatedProfiles for simulations) an sdtab table must be specified. The names but not the order of the columns is important. ID and TIME must be in the table. Other columns are optional. The example below shows an sdtab with all columns. The name of the file must start with "sdtab".\\ Example:
        \begin{verbatim}
$TABLE ID TIME MDV DV PRED IPRED RES IRES WRES IWRES CWRES
NOAPPEND FILE=sdtab2
        \end{verbatim}
    \item To be able to populate the IndividualEstimates (or the IndivParameters for simulations) entry a patab table must be specified. NOAPPEND must be used and the order of the columns is important. The first column must be ID, then comes all individual parameters followed by the ETAs (see below).
    
    Example:
        \begin{verbatim}
$TABLE ID CL V ETA_CL ETA_V NOAPPEND FILE=patab
        \end{verbatim}
    \item To be able to populate the Covariates entry in Simulation a cotab table must be specified. The names but not the order of the columns are important. ID and TIME must be in the table. Other columns are covariates. The name of the file must start with "cotab".\\ Example:
        \begin{verbatim}
$TABLE ID TIME WGT NOPRINT NOAPPEND FILE=cotab
        \end{verbatim}
    \item To be able to populate the RandomEffects entries all ETAs must be defined in the main code block (\$PK or \$PRED). One ETA should be defined per row with nothing more on the row (except for spaces) than one assignment.\newpage 
     Example:
        \begin{verbatim}
 ETA_CL = ETA(1)
 ETA_V = ETA(2)
        \end{verbatim}
    \item ETAs used for interoccasion variability must be handled like this:
        \begin{itemize}
            \item The \$OMEGA record describing the first occation is a normal omega BLOCK record and the following occasions are BLOCK SAME. Currently SAME(n) is not supported so each occasion must have its separate \$OMEGA record.\\ Example for three different occasions:
            \begin{verbatim}
$OMEGA BLOCK(4)
0.1 ; BOV_CL
0
0.1 ; BOV_V
0
0
0.1 ; BOV_KA
0
0
0
0.1 ; BOV_TL

$OMEGA BLOCK(4) SAME

$OMEGA BLOCK(4) SAME
            \end{verbatim}
            \item In the main code block (\$PK or \$PRED) the ETAs for the different occasions must be combined with the special \$ABBR syntax. Example for two occasions:
            \begin{verbatim}
$ABBR REPLACE ETA(OCC_ETA_BOV_CL) = ETA(5,9)
$ABBR REPLACE ETA(OCC_ETA_BOV_V) = ETA(6,10)
$ABBR REPLACE ETA(OCC_ETA_BOV_KA) = ETA(7,11)
$ABBR REPLACE ETA(OCC_ETA_BOV_TL) = ETA(8,12)
            \end{verbatim}
        \item In the main code block (\$PK or \$PRED) the combined ETAs must be assigned to variables using the correct name of the ETA and the correct name of the occasion. Example:
            \begin{verbatim}
ETA_BOV_CL = ETA(OCC_ETA_BOV_CL)
ETA_BOV_V = ETA(OCC_ETA_BOV_V)
ETA_BOV_KA = ETA(OCC_ETA_BOV_KA)
ETA_BOV_TL = ETA(OCC_ETA_BOV_TL)
            \end{verbatim}
        \end{itemize}
    \item If multiple DVs are to be used a DVID column must be set in the sdtab.
\end{itemize}
\subsection{Requirements on other files}
\begin{itemize}
    \item All NONMEM	 result files must have the same filename (excluding extension) as the .lst file. If not it is impossible to pick up the results from the .ext, .cov, .cor files etc.
\end{itemize}

\section{Known limitations}
\begin{itemize}
    \item The full xml file is stored in memory at once. This means that big results, for example big simulations, can potentially use a lot or all of the available memory.
    \item To be able to filter out dose rows for SimulatedProfiles and Residuals and MDV column must be included in the sdtab otherwise all rows will be added.
    \item PopulationParameters for simulations are assumed to be fixed for all individuals and occations thus not supporting \$PRIOR
    \item All parameters in a bootstrap will be on the var/cov scale regardless of the scale used in the model.
    \item Parameters with no labels will not generate any warnings for bootstrap when given on the sd/corr scale.
    \item Only the first \$PROBLEM will be added to the SO
    \item When using between occasion variability, which transalate to \$OMEGA BLOCK SAME in NONMEM, the OMEGAs for the SAME BLOCKs will be added to the PopulationParameters. These omegas will have their generic names (i.e. OMEGA\_2\_2)
\end{itemize}

\section{Output}

The resulting standard output object xml will be written to the file called <name of .lst file>.so\_xml.

The following entries will be populated:

\begin{itemize}
    \item RawResults - All used files and some selected not used files will be added. See the section on raw results for more info
    \item TaskInformation 
        \begin{itemize}
            \item Message - Errors, warnings etc will be put here. See section "information messages"
            \item RunTime - From the .lst file. Remember that RunTime only reflects the run time reported by NONMEM in the .lst file. No PsN runtime will be reported.
        \end{itemize}
    \item Estimation
        \begin{itemize}
            \item PopulationEstimates
            \begin{itemize}
                \item Bootstrap
                \begin{itemize}
                    \item Mean - From a PsN bootstrap\_results.csv file. All random effects will be on the var/cov scale.
                    \item Median - From a PsN bootstrap\_results.csv file. All random effects will be on the var/cov scale.
                \end{itemize}
                \item MLE - Results will be taken from the .ext file if present otherwise the .lst file. Parameters that are fixed and does not have a label will not be included. The estimated value
                    will be on the standard deviation/correlation scale or variance/covariance scale depending on which scale the initial estimate was on. 
            \end{itemize}
            \item PrecisionPopulationEstimates
            \begin{itemize}
                \item MLE
                \begin{itemize}
                    \item CovarianceMatrix - If the covariance step was successful. Results will be taken from the .cov file if present otherwise from the .lst file.
                    \item CorrelationMatrix - If the covariance step was successful. Results will be taken from the .cor file if present otherwise from the .lst file.
                    \item StandardError - Results will be taken from the .ext file if present otherwise the .lst file. Parameters that are fixed will get standard error NA.
                    \item RelativeStandardError - Results will be taken from the .ext file if present otherwise the .lst file. The RSE is calculated has the ratio between the standard error and the estimated value of the parameter and expressed as a percentage. Parameters that are fixed will have relative standard error set to NA.
                    \item ConditionNumber - Calculated from the eigenvalues in the .lst file if PRINT=E was present in the \$COV of the control stream.
                \end{itemize}
                \item Bootstrap
                \begin{itemize}
                    \item StandardError - From a PsN bootstrap\_results.csv file. Will give the SE for all parameters. 
                    \item AsymptoticCI - From a PsN bootstrap\_results.csv file. Will give the calculated percentiles at 2.5\% and 97.5\%
                    \item PercentilesCI - From a PsN bootstrap\_results.csv file. All random effects will be on var/cov form.
                \end{itemize}
            \end{itemize}
            \item IndividualEstimates
            \begin{itemize}
                \item Estimates
                \begin{itemize}
                    \item Median - Calculated from the patab if created
                    \item Mean - Calculated from the patab if created
                \end{itemize}
                \item RandomEffects
                \begin{itemize}
                    \item EffectMedian - Calculated from the patab if ETAs are named correctly
                    \item EffectMean - Calculated from the patab if ETAs are named correctly
                \end{itemize}
            \end{itemize}
            \item Residuals - Taken from the sdtab if present. If MDV column present only values for which MDV=0 will be added. If DVID column present in sdtab it will be added.
            \item Predictions - Taken from the sdtab if present. If DVID column present in sdtab it will be added.
            \item OFMeasures
                \begin{itemize}
                    \item Deviance - This is the NONMEM ofv value taken from the .lst file
                \end{itemize}
        \end{itemize}
    \item Simulation 
        \begin{itemize}
            \item SimulationBlock    
            \begin{itemize}
                \item SimulatedProfiles - Taken from the first table that contains all of ID, TIME and DV. If MDV column is present only DVs for which MDV=0 will be added. DVID will be set according to the DVID column in the same table file. If DVID column is not present DVID will be set to 1.
                \item IndivParameters - Taken from patab
                \item Covariates - Taken from cotab
                \item PopulationParameters - Created from the initial estimates of the parameters from the .lst file and the maxímum span of TIME from the first table that contains TIME.
                \item Dosing - Created from the original dataset. Can currently only handle the AMT column.
            \end{itemize}
        \end{itemize}
\end{itemize}

\subsection{vpc}
If nmoutput2so was run with the -rundir option in a vpc directory the original data table and the simulations table will be added.
The original data table to ModelDiagnostic/DiagnosticStructuralModel/IndivObservationPrediction and the simulations table to
Simulation/SimulationBlock/SimulatedProfiles (each replicate in its own table).


\subsection{Raw results}
Some special raw results files in the RawResults section of the SO will have a specific oid set. This makes it easier to programatically find the files. Here follows a list of these files and oids:

\begin{tabular}{ l l}
      oid & file \\
    \hline
      PsN\_bootstrap\_results & bootstrap\_results.csv \\
      PsN\_bootstrap\_included\_individuals & included\_individuals1.csv \\
      PsN\_bootstrap\_raw\_results & raw\_results\_<model>.csv \\
      PsN\_SSE\_results & sse\_results.csv \\
      PsN\_SSE\_raw\_results & raw\_results\_<model>.csv \\
      PsN\_VPC\_results & vpc\_results.csv \\
      PsN\_VPC\_vpctab & vpctab<num> \\
\end{tabular}



\subsection{Information messages}
A set of tool specific information messages are added in TaskInformation. They are listed table below:

\begin{itemize}
    \item \textbf{estimation\_successful} \\ Set to 1 if the estimation was successful otherwise set to 0.
    \item \textbf{covariance\_step\_run} \\ Set to 1 if the covariance step was run otherwise set to 0. The information is taken from the .lst file
    \item \textbf{covariance\_step\_successful} \\ Set to 1 if the covariance step was successful
    \item \textbf{covariance\_step\_warnings} \\ Set to 1 if any of "R MATRIX ALGORITHMICALLY SINGULAR" or "S MATRIX ALGORITHMICALLY SINGULAR" is present in the .lst file. Set to 0 otherwise.
    \item \textbf{rounding\_errors} \\ Set to 1 if "ROUNDING ERRORS" is present in the .lst file.  Set to 0 otherwise.
    \item \textbf{hessian\_reset} \\ Set to the number of appearances of "RESET HESSIAN" in the .lst file
    \item \textbf{zero\_gradients} \\ Set to the number of zero gradients in the "GRADIENT" lists of the .lst file
    \item \textbf{final\_zero\_gradients} \\ Number of zero gradients in the final iteration in the .lst file.
    \item \textbf{estimate\_near\_boundary} \\ Set to 1 if any of "ESTIMATE OF THETA IS NEAR THE BOUNDARY" or "PARAMETER ESTIMATE IS NEAR ITS BOUNDARY" is present in the .lst file. Set to 0 otherwise.
    \item \textbf{significant\_digits} \\ NO. OF SIG. DIGITS IN FINAL EST. from .lst file
    \item \textbf{s\_matrix\_singular} \\ Set to 1 if "S MATRIX ALGORITHMICALLY SINGULAR" present in .lst file. Set to 0 otherwise.
    \item \textbf{condition\_number} \\ If the eigenvalues of the correlation matrix was printed to the .lst file (i.e. option PRINT=E was given to \$COV) nnmoutput2so calculates the condition number. 
\end{itemize}


\section{Not supported SO elements}

This section lists all the elements of the SO that is not supported and that will not be added to the xml file.

\begin{verbatim}
SO/
    SOBlock/
        ToolSettings
        TaskInformation/
            OutputFilePath
            NumberChains
            NumberIterations
        Estimation/
            PopulationEstimates/
                Bayesian
            PrecisionPopulationEstimates/
                MLE/
                    FIM
                    AsymptoticCI
                Bayesian
                LLP
                SIR
                MultiDimLLP
                EtaShrinkage
                EpsShrinkage
            IndividualEstimates/
                Estimates/
                    Mode
                    Samples
                RandomEffects/
                    EffectMode
                    Samples
            PrecisionIndividualEstimates
            Residuals/
                PD and NPDE columns
            OFMeasures/
                Likelihood
                LogLikelihood
                ToolObjFunction
                IndividualContribToLL
                InformationCriteria
        Simulation/
            OriginalDataset
            SimulationBlock/
                RandomEffects
                RawResultsFile
\end{verbatim}

\references

\end{document}
