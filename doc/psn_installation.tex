\documentclass[a4paper,12pt]{article}
\title{PsN installation\\ \vspace{2 mm} {\large PsN 3.6.2}}
\date{2013-05-28}

\usepackage[utf8]{inputenc}
\usepackage{verbatim}
\usepackage{longtable}
\begin{document}

\maketitle

Before you install PsN, you should make sure that the following programs are installed on your computer:
\begin{itemize}
	\item NONMEM 

Please note that if you intend to run PsN with NMQual8 the nm72.xml file must be slightly modified (to expect .mod instead of .ctl as the control stream suffix) before NONMEM installation, see document psn\_configuration.pdf, otherwise PsN's NMQual8 support will not work.

You must verify that all NONMEM installations you intend to use can be run directly via the nmfe/NMQual scripts and that they produce complete output files. If you plan to use the parallelization features of NONMEM7.2 you must also verify that you can run NONMEM in parallel directly with nmfe72.

If running NONMEM with nmfe/NMQual does not work then PsN will not work. 
	\item Perl5
	\item Perl modules Math::Random, Statistics::Distributions, Archive::Zip and File::Copy::Recursive (installation instructions below). PsN can be run without Statistics::Distributions, Archive::Zip and File::Copy::Recursive but some scm and vpc features and a scm bugfix will not be available.
\end{itemize}

\section{Install Perl modules Math::Random, Statistics::Distributions, Archive:Zip, File::Copy::Recursive}
PsN requires the Perl module Math::Random to be installed on your computer. The other modules are very useful, but not absolutely required. The following paragraphs describe the setup procedure for different operating systems. 
\subsection{Windows (ActivePerl)}
For this procedure to work you must be connected to the internet, because the package is downloaded from the web. Pay attention to any warning messages from Windows that talk about internet connection problems, such problems must be solved before doing the installations.

Open a DOS window and run the commands (press Enter after each command):
\begin{verbatim}
ppm install math-random
ppm install statistics-distributions
ppm install archive-zip
ppm install file-copy-recursive
\end{verbatim}
An alternative way, using the GUI:

\begin{enumerate}
\item Start the Perl package manager (Start$>>$All programs$>>$ Active Perl$>>$Perl Package Manager).
\item Type math-random in the search box at the top of the Perl package manager (PPM) window (note the minus instead of the double colons). 
\item Right click on the Math-Random package (select the package without suffix after Math-Random, you can increase the width of the column showing the package names to be sure) and select 'Install' (could be under menu Actions). Sometimes Math::Random is already installed by default, then you do not have to reinstall it, proceed to step 4.
\item Type statistics-distributions in the search box at the top of the Perl package manager (PPM) window (note the minus instead of the double colons). 
\item Right click on the Statistics-Distributions package (select the package without suffix after Statistics-Distributions) and select 'Install' (could be under menu Actions)
\item Type archive-zip in the search box at the top of the Perl package manager (PPM) window (note the minus instead of the double colons). 
\item Right click on the Archive-Zip package (select the package without suffix) and select 'Install' (could be under menu Actions)
\item Type file-copy-recursive in the search box at the top of the Perl package manager (PPM) window (note the minus instead of the double colons). 
\item Right click on the File-Copy-Recursive package (select the package without suffix) and select 'Install' (could be under menu Actions)
\item Click on 'Run marked actions' (can be a green arrow in the menu) or press [CTRL+ENTER]
\end{enumerate}


\subsection{Linux}

\begin{enumerate}
\item Download the Math-Random package from the CPAN website 
\item Unpack the archive to a temporary directory of your choice 
\item Open a command line window and move to the directory chosen in step 2 
\item Run: \verb|perl Makefile.PL|
\item Run: \verb|make| 
\item Run: \verb|sudo make install| (the install script will automatically locate the Perl module directory and put Math::Random in a suitable place) 
\item Delete the temporary directory from step 2.
\end{enumerate}

The other modules are installed in the same way.

\section{PsN installation instructions}

\begin{enumerate}
\item Unpack the file you downloaded from psn.sourceforge.net. It will create a directory called PsN-Source.
\item Run the installation script from within PsN-Source. If you are running windows and have ActiveState ActivePerl installed you should be able to double-click on setup.pl. Otherwise open a command line window (Windows Start-$>$Run, type 'cmd'), go to the PsN-Source directory and type: 
\begin{verbatim}
perl setup.pl
\end{verbatim}
Unix users should open their favorite terminal, go to the PsN-Source directory and type:
\begin{verbatim}
perl setup.pl
\end{verbatim}
\item Answer the questions on screen. The default is probably the best for most users. It is really, really recommended to let the script generate psn.conf for you, unless you are experienced in installing and configuring PsN.
If you cannot install PsN where the install script suggests and you wish to use PsN in your own perl scripts, you must make sure that the directory where you installed the PsN core and toolkit is in Perl's include path. For convenience you should also check that the directory where the utilities are installed is in your search path. 
\item If you did not let the installation script create a configuration file psn.conf for you, or if you are running NONMEM via NMQual or on a cluster, you need to edit psn.conf in the PsN installation directory. The document psn\_configuration.pdf in PsN-Source/doc describes how to make correct settings in the psn.conf file. The help document is also found on the PsN website under Documentation.
\item When the installation is done you can safely remove the PsN-Source directory if you like. 
\end{enumerate}

\end{document}
