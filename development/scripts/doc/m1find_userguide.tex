\documentclass[a4wide,12pt]{article}
%\setlength{\marginparwidth}{0pt}%35
%\setlength{\marginparsep}{0pt}%?
%\setlength{\evensidemargin}{0pt}
%\setlength{\oddsidemargin}{0pt}
\usepackage{lmodern}
\usepackage[utf8]{inputenc}
\usepackage[T1]{fontenc}
\usepackage{textcomp}
\usepackage{verbatim}
\usepackage{enumitem}
\usepackage{longtable}
\usepackage{alltt}
\usepackage{ifthen}
%\usepackage[backend=biber, maxbibnames=99, defernumbers]{biblatex}
%\addbibresource{PsN.bib}
% Reduce the size of the underscore
\usepackage{relsize}
\renewcommand{\_}{\textscale{.7}{\textunderscore}}

\newcommand{\revisiondate}[1]{
\date{#1}
}
\newcommand{\guidetitle}[2]{
\title{#1\\ }
\date{Revised #2}
}


\newcommand{\doctitle}[2]{
\title{#1}
\date{#2}
}

\newcommand{\optname}[1]{\item{{\bfseries\texttt-#1}\newline}}
\newcommand{\optdefault}[2]{\item{{\bfseries\texttt-#1}{\mbox{ = \it #2}}\newline}}

\guidetitle{m1find user guide}{2016-03-16}

\begin{document}

\maketitle
\newcommand{\guidetoolname}{m1find}


\section{Introduction}
\subsection{What is the m1 subfolder}
All PsN programs (except execute) that run NONMEM will create a m1 subfolder in
the main run directory. The m1 subfolder contains intermediate files,
such as bootstrapped data sets and lst-files from individual NONMEM runs, before
results are summarized. When a run is finished, the m1 folder is not cleaned
away by PsN, even though the contents are usually not needed.
%, because \emph{sometimes} the contents are still needed.

In most situations,
the m1 contents are just a nuisance, and the collected m1 subfolders
are a great burden to our file storage and backup system. The more files on
the system, the higher risk of file system or backup failure
and very long recovery times before we can work normally again.
About a third of all files on our system is old m1 folders, 
and if all
m1 folders were removed, we would not have a backup problem anymore.
In a few situations the m1 folders needs to be zipped and kept,
and those situations
are listed below.

\subsection{m1find}
The program m1find is a simple method to remove or zip m1 folders. 
The names of all removed folders are logged in a file in the user's home directory,
and can be easily retrived from backup by the system administrator in case the
user realizes the folder is still needed.

\section{What to do with m1}
\subsection{When to keep m1 untouched}
The \emph{only} situation when m1 should be kept untouched is when it is part of a job that
is running on the cluster \emph{now}, i.e. something that shows
up in squeue. This is at most 20 m1 folders per user, unless you have many
models running in the lowprio queue.

{\bf All} other m1 subfolders should be zipped or, preferably, removed, and m1find can do it for you.

\subsection{When to zip m1}
\begin{itemize}
\item you are not done with binning in a vpc, then m1 contains files needed to redo the binning
\item you want to use -add\_retries in sse on previously simulated datasets \emph{and} simulated data sets take a
long time to regenerate. If data sets are quick to generate, then remove m1.
\item you have an interrupted bootstrap, where some but not all samples have lst-file in m1.
\emph{Note:} if you have a complete raw\_results file for
bootstrap you should remove m1, because you can recompute percentiles etc with different exclusion criteria (estimation\_terminated
and such options) using only the raw\_results file.
\end{itemize}
PsN will automatically unzip a zipped m1 folder if you are restarting a run in
an existing PsN run directory, 
so you do not have to unzip anything manually for example if you run sse with -add\_retries,
or redo the binning of a vpc.

\subsection{When to remove m1}
\begin{itemize}
\item Always except in cases listed above
\end{itemize}


\section{How to use m1find}

The command structure is\\
m1find ``what to do'' ``where'',\\
for example\\
m1find remove course\_work


\subsubsection*{where}
``where'' is the name of one or more directories, for example course\_work, Project\_1 or bootstrap\_dir2. Multiple directories
are listed with a space in between. A single dot (.) means the current working directory.
Unix wildcards can be used to easily input multiple directory names, for example sse* which will match
all directories in the current working directory that start with sse.
The program will find all m1 directories \emph{anywhere inside} the given directories,
it can search through multiple directory levels. See examples further down.

\subsubsection*{what to do}
``what to do'' is either list, zip, remove or ask.
\begin{description}
\item[list] All m1 directories anywhere inside the given folder(s) will be found and listed on screen.
         No zipping or removing will take place. 
\item[zip] All m1 directories anywhere inside the given folder(s) will be found and listed on screen.
         The user will then have the option to zip and remove all.
\item[remove] All m1 directories and m1.zip files anywhere inside the given folder(s) will be found and listed on screen.
         The user will have the option to remove all.
         All removed files will be logged in the file removed\_by\_m1find in your home directory.
\item[ask] All m1 directories anywhere inside the given folder(s) will be found and presented one by one.
         The user has the option to zip,remove or skip each directory separately.
\end{description}

\subsubsection*{Do not disturb other cluster users}
To be on the safe side, always
start an interactive slurm session before running m1find,
to reduce the load on the login node.
\begin{verbatim}
srun --pty bash
\end{verbatim}

\noindent After you are done with m1find, end the session with command \verb|exit|.

\subsection{Examples}

\subsubsection*{List all my m1 folders}
Use cd to go to your home directory and then run
\begin{verbatim}
m1find list .
\end{verbatim}


\subsubsection*{Remove all m1 in project folder ProjectA}

\begin{verbatim}
m1find remove path/to/ProjectA
\end{verbatim}
Remember that names of all removed m1 will be stored in a file in your home directory.

\subsubsection*{Keep a few m1 folders in a project directory}
I have 30 bootstrap folders in my project folder, where bootstrap\_dir 1-28 are done and
bootstrap\_dir29 and bootstrap\_dir30 are running now:
Remove all m1 in bootstrap\_dir with number starting with 1 (1, 10-19):
\begin{verbatim}
m1find remove bootstrap_dir1*
\end{verbatim}
Interactively go through all remaining m1, (want to remove all except m1 in bootstrap\_dir29 and bootstrap\_dir30): 
\begin{verbatim}
m1find ask bootstrap_dir*
\end{verbatim}


\section{Additional options}
\begin{description}
\optname{no-interactive} Skip confirmation question before removing or zipping.
\optname{nm\_run} Find NM\_run directories. Will be ignored with 'zip'.
\optname{no-findm1} Only for use with -nm\_run: Skip search for m1 and m1.zip
\optdefault{logdir}{directory} Default user's home directory. The directory to write
log file removed\_by\_m1find  
\optname{h} Print help text and exit.
\optname{help} Print help text and exit.
\end{description}

\end{document}
