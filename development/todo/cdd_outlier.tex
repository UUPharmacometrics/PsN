\documentclass[a4wide,12pt]{article}
%\setlength{\marginparwidth}{0pt}%35
%\setlength{\marginparsep}{0pt}%?
%\setlength{\evensidemargin}{0pt}
%\setlength{\oddsidemargin}{0pt}
\usepackage{lmodern}
\usepackage[utf8]{inputenc}
\usepackage[T1]{fontenc}
\usepackage{textcomp}
\usepackage{verbatim}
\usepackage{enumitem}
\usepackage{longtable}
\usepackage{alltt}
\usepackage{ifthen}
%\usepackage[backend=biber, maxbibnames=99, defernumbers]{biblatex}
%\addbibresource{PsN.bib}
% Reduce the size of the underscore
\usepackage{relsize}
\renewcommand{\_}{\textscale{.7}{\textunderscore}}

\newcommand{\revisiondate}[1]{
\date{#1}
}
\newcommand{\guidetitle}[2]{
\title{#1\\ }
\date{Revised #2}
}

\guidetitle{outlier diagnostics with cdd}{2016-04-14}


\begin{document}

\maketitle
\newcommand{\guidetoolname}{new vpc}


\section{Introduction}
Influential individual diagnostics using cdd

\subsubsection*{CDD today Cook score}
Overall Cook score per cdd dataset (with one individual
removed in each). The greater the influence of an individual the greater the
Cook score. Cook score is
$\sqrt{\Delta params^T\cdot COV_{orig}^{-1}\Delta params}$ where $\Delta params$
is cdd data set estimated parameter vector minus
original data set parameter vector, and $COV_{orig}$ 
is original data set covariance matrix.

\subsubsection*{CDD today Covariance ratio}
Covariance ratio is output per cdd dataset. Covariance ratio for
individual $i$ is $\sqrt{\frac{determinant(COV_{i})}{determinant(COV_{orig})}}$
where $COV_{orig}$ is 
is original data set covariance matrix, $COV_{i}$ is covariance matrix
for data set with individual $i$ excluded.

If covariance ratio is low then problematic individual, get much lower SE:s
when individual removed. If cov ratio close to 1 then not influential
individual. If covariance ratio is high then individual has much information,
get much higher SE:s when individual is removed.

%\begin{verbatim}
%vpcmixture -samples=200 run1.mod
%\end{verbatim}

\subsubsection*{Possible extension Cook score}
Compute Cook scores per parameter $k$ and individual $i$: 
$\frac{\Delta_i \theta_k}{SE_{\theta k,orig}}$ 
where $\Delta_i \theta_k$ is change in estimate of $\theta_k$ between original
estimation and estimation with individual $i$ removed from dataset.

\subsubsection*{Possible extension Cov ratio}
Compute Covariance ratio per parameter: 
$\frac{SE_{\theta k,cdd-i}}{SE_{\theta k,orig}}$, 
ratio of SE of $\theta_k$ from estimation with and without individual $i$
in dataset.

\section{new Procedure }
\begin{enumerate}
\item run regular cdd with case\_column=ID
\item For each ID $j$, compute $\Delta iOFV_{j}$:  
\begin{enumerate}
\item From original estimation
  phi-file get $origOFV_{no-j}$: total original OFV minus original iOFV
  for individual $j$.
\item From raw\_results file, the row for
  estimation of cdd dataset with individual $j$ removed get total OFV
  from column ofv: $cddOFV_{no-j}$
\item Let $\Delta iOFV_{j} = origOFV_{no-j} - cddOFV_{no-j}$ which should
  always be non-negative.
\end{enumerate}
\item For each ID $j$, compute $\Delta iOFV_{xv-j}$:  
\begin{enumerate}
\item Use estimated parameters from estimation of cdd dataset with
  individual $j$ removed, run MAXEVAL=0 on only individual $j$ data to
  get $iOFV_{j,xv}$
\item From original estimation
  phi-file get $iOFV_{j,original}$: original iOFV for individual $j$
  when all data was used for parameter estimation.
\item Let $\Delta iOFV_{xv-j} = iOFV_{j,xv}-iOFV_{j,original}$. This
  should be non-negative since original iOFV should be smaller since
  since observation $i$
  was included in estimation of parameters.
\end{enumerate}
\item Plot $\Delta iOFV_{xv-j}$ against $\Delta iOFV_{j}$.
\end{enumerate}

\end{document}

